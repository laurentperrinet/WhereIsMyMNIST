% !TEX root = paper.tex
% !TEX encoding = UTF-8 Unicode
% -*- coding: UTF-8; -*-
% vim: set fenc=utf-8
% !TEX spellcheck = en-US
\section{Discussion}
\label{sec:discussion}
%\subsection{Summary}
%## Main results:

In summary, we propose a visuo-motor action-selection model that implements a focal accuracy-seeking policy across the image. It relies on an interpretation of the Information Gain metric as a difference between central and peripheric accuracy processing.
Each accuracy is predicted through separate processing pathways, namely the ``What'' pathway for the central pixels and the ``Where'' pathway for the periphery. The comparison of each accuracies amounts either to select a saccade or to keep the eye focused at the center (so as to identify the label).
The predicted accuracy map has, in our case, the role of a value-based action selection map, as it is the case in model-free reinforcement learning. However, it also owns a probabilistic interpretation that may be combined with concurrent accuracy predictions (such as the one done through the ``what'' pathway) to bring out more elaborate decision making, such as for instance the inhibition of return. 

%- interpretation
%- An effective decoding scheme with strong bandwidth reduction
%- Information-gain based selection of action 

%- A sub-linear object detection for image processing:
%- A full log-polar processing pathway (from early vision toward action selection)
%- Sequential info gain converges to zero: in practice 2-3 saccades are enough
%- Ready for up-scaling

%- Object identity-based monitoring of action
%- Dorsal = ''actor'' (where to look next?)
%- Ventral = ''critic'' (for what to see?)

The approach is also energy-efficient. It encompasses a full log-polar processing pathway which preserves the strong compression rate performed by retina and V1 encoding up to the action selection level. It finally provides an effective sub-linear decoding scheme, that may allow to detect object in large visual environments at little cost. This should be envisioned when the computing resources are under constraint, such as for drones or mobile robots. 


%\subsection{Limits}

% deep learning proves that one solution exists, but what is the visual substrate actually used? 

%## Limits and Open questions
%- Importance of centering objects:
%- Central object referential
%- log polar scale/rotation invariance
%- (feedback) prediction
%- Information Gain-based décision :
%- Sequential info gain converges to zero: in practice 2-3 saccades are enough
%- Pursuit vs. saccade.
%- Maximizing info gain on multiple targets/ddls.
%- Overt/covert attention
%- Inhibition of return

One crucial aspect of vision highlighted by our model is the importance of centering objects in recognition. Despite the robust translation invariance observed on the ``What'' pathway, there is small radius of 2-3 pixels around the target center that needs to be respected to maximize the classification accuracy. This relates to the idea of finding an absolute referential for an object, for which the recognition is easier. If the center of fixation is fixed, the logpolar encoding of an object class shows invariance to both rotation and scale \citep{Traver10}. Incorporating this scale and rotation invariance may thus be considered to extend the recognition capabilities of the model.

More elaborate image categorization, such as the ones performed on the Imagenet dataset by deep convolutional nets [REF], should also be envisioned.
%By preserving a probabilistic interpretation in bio-realistic action selection, 
Real visual scan path over images could then be used to provide realistic priors over action selection maps.  %
%It may indeed be possible to consider , and 
Identified regions of interest may then be compared with the baseline bottom-up approaches, such as the low-level feature-based saliency maps~\citep{Itti01}. 

Finally, our model relies on a strong idealization, assuming the existence of a probabilistic representation of action achievement over large action spaces in the brain. 
The presence of many targets in a scene may be addressed %which amounts to sequentially select targets, in combination with implementing an inhibition of return mechanism. 
through maximizing the Information Gain over concurrent future accuracies. %needs to be envisioned with a more refined probabilistic framework, including mutual exclusion over overt and covert targets.
Still, how the brain may combine and integrate those various probabilities is an open question, that resorts to the classical binding problem. %: How is it possible to meaningfully combine independently extracted features.
%\subsection{Perspectives}
