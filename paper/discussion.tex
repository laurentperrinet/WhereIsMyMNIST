% !TEX root = paper.tex
% !TEX encoding = UTF-8 Unicode
% -*- coding: UTF-8; -*-
% vim: set fenc=utf-8
% !TEX spellcheck = en-US
\section{Discussion}
\label{sec:discussion}
\subsection{Summary}
The predicted accuracy map has, in our case, the role of a value-based action selection map, as it is the case in model-free reinforcement learning. However, it also owns a probabilistic interpretation that may be combined with other accuracy predictions (such as the one done through the ``what'' pathway) which potentially explains more elaborate decision making, such as inhibition of return. The approach is also energy-efficient as it includes the strong compression rate performed by retina and V1 encoding, which is preserved up to the action selection level. When energy and computing power are at stake, as it is the case in bio-inspired robotics, it may thus be relevant to envision our implementation. By preserving a probabilistic interpretation in bio-realistic action selection, we also allow for a principled use of human visual scan path over images, such as is generally modeled as low-level feature-based saliency maps~\citep{Itti01}. It may indeed be possible to consider the actual action selection as implementing focal accuracy-seeking policy across the image, and learn the actual saccade path as the response of a pre-trained accuracy prediction. Identified regions of interest may then be compared with the baseline bottom-up approaches.

\subsection{Limits}

% deep learning proves that one solution exists, but what is the visual substrate actually used? 

From the modeling side, our model still relies on a strong idealization, assuming the existence of a probabilistic representation of action achievement over large action spaces in the brain. Similarly, how the brain may combine and integrate various probabilities is still an open question, that resorts to the classical binding problem. %: How is it possible to meaningfully combine independently extracted features.
At last, the presence of many targets in a scene should also be addressed by the model, which resorts to sequentially select targets, in combination with a concrete implementation of an inhibition of return mechanism.

%\subsection{Perspectives}
