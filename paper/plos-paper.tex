%!TeX TS-program = pdflatex
%!TeX encoding = UTF-8 Unicode
%!TeX spellcheck = en-US
%!BIB TS-program = bibtex
% -*- coding: UTF-8; -*-
% vim: set fenc=utf-8
%
% rev 0 = https://www.biorxiv.org/content/10.1101/725879v1.full.pdf #######################################################################################################################
%%% reviews from PLoS:
%%
%%On 23 Dec 2019, at 08:58, PLOS Computational Biology <em@editorialmanager.com> wrote:
%%
%%Dear Dr. Perrinet,
%%
%%Thank you very much for submitting your manuscript "A dual foveal-peripheral visual processing model implements efficient saccade selection" (PCOMPBIOL-D-19-01274) for consideration at PLOS Computational Biology. As with all papers peer reviewed by the journal, your manuscript was reviewed by members of the editorial board and by several independent peer reviewers. Based on the reports, I regret to inform you that we will not be pursuing this manuscript for publication at PLOS Computational Biology.
%%
%%There are several technical issues and suggestions raised by the reviewers, which I would believe could be fixed in a major revision. Unfortunately, however, as reviewer #2 points out, there is some lack of novelty, which precludes this manuscript from being considered further for PLoS Computational Biology. Nonetheless, I hope that the reviewers' comments will be helpful for submission elsewhere, e.g., PLoS One, whose focus is on technically sound work but less on novelty.
%%
%%The reviews are attached below this email, and we hope you will find them helpful if you decide to revise the manuscript for submission elsewhere. I am sorry that we cannot be more positive on this occasion. We very much appreciate your wish to present your work in one of PLOS's Open Access publications.
%%
%%Thank you for your support, and I hope that you will consider PLOS Computational Biology for other submissions in the future.
%%
%%
%%Sincerely,
%%
%%Wolfgang Einhäuser
%%Deputy Editor
%%PLOS Computational Biology
%%
%%
%%
%%**************************************
%%Reviewer's Responses to Questions
%%
%%<b>Comments to the Authors: <br/>Please note here if the review is uploaded as an attachment.</b>
%%
%%Reviewer #1: The manuscript proposes a dual foveal-peripheral visual processing model that that classifies MNIST figures over a noisy background. The model predicts input locations with high expected prediction accuracy via a "Where"-Pathway which extracts responses of linear orientation filters spaced retinotopically (higher frequencies closer to the fovea) and linearly decodes a predicted accuracy map from those filters in the same spacing. Then a high-resolution version of the input at that location of highest predicted accuracy is fed to the "What"-pathway that tries to predict which digit has been shown.
%%The model is trained on simulated data. Model performance is evaluated and analyzed depending on contrast, target eccentricity and number of saccades performed, as well as the dependency on hyper parameters.
%%
%%# Novelty and related works
%%The main novelty in this manuscript is the use of a periphery with more human-like sensitivity compared to, e.g., ref [14] which uses a periphery with constant resolution. I very much like this approach of using a more human-like sensitivity in a foveated detection model and I think it has the potential to be an interesting contribution to the field. The paper gives a good overview of the related works.
%%
%%# Claims and results
%%The main claim stated in the abstract is that the presented approach is beneficial compared to mainstream computer vision. I would like to ask the authors to state this claim more precisely (at least in the main text): Beneficial in what way exactly?Processing time? Computational resources? Robustness? Human-likeness? Other things I didn't think about?
%%
%%Additionally, I think this claim, however it is meant exactly, might need additional evidence to be well supported. The authors discuss the the relation with other models (line 555), but this section mainly discusses where the presented model is different from related models. There is no quantitative comparison to other models which I would consider necessary to show the advantages of the presented model. More precisely:
%%
%%- I believe the authors that their approach is beneficial compared to, e.g., a classic fully convolutional neural network, however since the presented model always needs at least two sequential processings (first where, then what) this is not completely obvious. Only if the "where" question can be solved with substantially less resources than the local "what" question, the two pathways are really an advantage. Most likely this is the case, but it should be discussed.
%%
%%- Less obvious but, as I think, much more interesting is, whether the presented model also has advantages compared to other applications of attention in computer vision. The authors put emphasis on the log-polar processing of the input space for the "where"-pathway as they say this is closer to what we find in the visual system compared to other models with implicit attention in computer vision. I agree that this is the main novelty of the manuscript. However, I'm missing any discussion of how this affects the model. A comparison with a model like ref [14], which seems to be most closely related since it also detects MNIST digits but uses a periphery with constant resolution, would be required to discuss the benefit of the log-polar representation. Is the model more efficient (e.g., when it comes to corrective saccades or with respect to parameters)? Is the performance better in relevant cases? Is it making errors that are more human like than a model like [14]? I think here
%%the authors are missing a chance to substantially increase the impact of the manuscript.
%%
%%# Clarity
%%The manuscript includes most necessary information to understand the reasoning, the model and the experiments. I'm especially happy to see that all code and data has been released and thank the authors for that.
%%
%%I think the clarity of the presentation could be improved in some cases:
%%- A main part of the model is the output of the "where"-pathway which is a prediction of accuracy. Therefore there are different notions of accuracy in the paper and it is sometimes hard to deduce whether the authors are at a certain place talking about
%%- actual accuracy when evaluated at a certain location on a certain input (either 0 or 1).
%%- expected accuracy at a certain location when averaging over all possible digits with the results of figure 3. I think often this is what "actual accuracy" or "ground truth" is used to mean in the manuscript.
%%- predicted accuracy of a certain location as predicted by the "where"-pathway.
%%
%%- I found the section "Peripheral vision: from log-polar feature vectors to log-polar action maps" (starting at line 338) a bit hard to understand. It might benefit from some restructuring.
%%- the section introduces the log-polar filters and the accuracy map and only at the very end I understood how everything fits together. I think it would be better to first simply describe the architecture of the pathway (linear filters with retinotopic distribution, linear operation and sigmoid to predict local accuracies) and then go into more detail on why the different parts where designed in a certain way. Also, maybe a figure that shows more architectural details of the "where"-pathway than figure 2 might be helpful.
%%- It would be very helpful to have at least a little bit more details on what the log-polar oriented filters are. Right now there is only a reference to [28].
%%- I don't really understand lines 391-400.
%%- According to line 409 the loss function of the classifier is a KL-divergence. However, neither the output of the accuracy map nor the retinotopic vector a are distributions. As far as I understood the, they encode a probabilities for each location. So I don't really see how a KL-divergence could be applicable here.
%%
%%- I have problems understanding Figure 7:
%%- Accuracy should be the dependent variable, correct? Then I would expect it to be on the y axis, not the x axis
%%- What are the horizontal lines? What do the different colors mean? Why is there sometimes a little marker at the border between the two colors?
%%- does the figure include dependency on the hyperparameters of the log polar filters, i.e., number of eccentricities, number of orientations, ...? If not: I think this would be very important to check
%%
%%- I'm not sure I understand why "recurrent attention is at odd with the functioning of biological systems" (line 565)
%%
%%# Other ideas
%%I want to list some other ideas that came to my mind while reviewing the paper. I want to emphasize that I don't consider these points relevant for the acceptance of the paper
%%- It might be interesting to train both pathways jointly
%%- It's interesting that the "what"-pathway is highly nonlinear but the "where"-pathway is linear. It might be interesting to check whether with simple nonlinearities the performance of the model could be increased.
%%
%%Reviewer #2: This paper proposes a foveated visual search model. Both the foveal and peripheral processing of the model are based on deep neural networks. The model is experimentally evaluated on a search task for handwritten digits on cluttered backgrounds.
%%
%%The main claim of the paper is that it is “the first case of a bio-realistic log-polar implementation of an active vision framework.” (line 556). However, this is not the case. The “foveated object detector” (FOD) by Akbas and Eckstein (PLOS Computational Biology, 2017) proposed a very similar model to that of the current manuscript and in fact, the FOD is more general in several ways: (i) it can handle real-world objects (not just simple digits), (ii) scale-variance of objects are handled, (iii) a comparison with its sliding-window counterpart is presented, (iv) the case where “the number of targets is unknown” is handled. In general, the current manuscript does a poor job of reviewing the literature. For example, another very similar model (Target Acquisition Model (TAM) by G. Zelinsky) is not mentioned at all. The related work presented by Akbas and Eckstein (2017) is more general and comprehensive.
%%
%%The main weakness of the manuscript is its lack of originality, which -- I think - caused by not reviewing the prior art properly. I recommend to re-submit after addressing the lack-of-originality concern and improving the manuscript along the issues/questions I raise below.
%%
%%At the beginning of the Introduction, authors use both “image processing” and “computer vision” to refer to the same thing. I think it would be better to stick with just “computer vision”.
%%
%%In the Introduction, references should be provided for the following statements: “fovea (a disk of about 6 degrees of diameter…” and “they take about 200 ms to initiate, last about 200 ms and … 600 degrees per second.”
%%
%%In line 86, authors say “can be found in [10,11,16] that will be compared further on with our approach.” However, this comparison is only in the form of a discussion (lines 555-585). Upon reading line 86, I expected to see experimental comparisons. In fact, no actual (experimental) comparisons with other models are done. As a reader, at the least, I expected a comparison with a baseline model which implements the classical sliding-window method.
%%
%%In line 81, authors claim that “sequential implementations have not been shown effective enough to overtake static object search methods.” However, there are at least three models (FOD, TAM and Infomax) which were shown to be superior to (in some performance metric) classical sliding-window approach. So, the authors should revise their statement.
%%
%%In general, the use of citations in text is not professional. There are usages like “study of [19]”, “in [10,11]”, “by [18]”, etc. These should be corrected.
%%
%%In line 150, the MNIST dataset appears for the first time, where it should be cited.
%%
%%The first sentence in the “Active inference” sub-section reads “This kind of reasoning…” (line 177). Which kind of reasoning? Because this is the start of a section, whatever “This” is referring to should be made more clear. This sentence should be more stand-alone.
%%
%%In line 178, I could not understand: “... the cause of a visual scene is couple made of a viewpoint …”. Please revise.
%%
%%Throughout the text, the variable “x” is described in several different ways: (i) state of sensor, (ii) partial view of the scene, (iii) visual field, and (iv) visual sample. This might be confusing to the reader. Please be consistent.
%%
%%In line 237: “predict for all” -> predict what? Not clear.
%%
%%Throughout the text, both “Fig” and “fig” are used. Please be consistent and follow journal’s style rules.
%%
%%In Figure 2 part (D), a stopping condition is explicitly given. However, this rule is not explicitly mentioned in the text. The relevant part is the “Concurrent action selection” on page 16. This section should be improved for clarity. And, explicit connections should be drawn to Figure 2.
%%
%%The paragraph about generating “Full-scale image” (line 279-282) should be improved. I could not understand how it is done.
%%
%%Authors claim the foveal processing has some translational invariance. However, they do not mention the use of any pooling layers in the convolutional net. Are there any pooling layers? If not, how does the conv net achieve translation invariance? Would not it be better to use max-pooling? The architecture of the conv net is not given in the paper. And from the reference [26], I could not find it. Care must be taken when formatting the references. If the reference is a web-page, its URL must be given along with its last accessed date.
%%
%%Lines 328-333 should be improved for clarity.
%%
%%In line 471, authors say “a much lesser cost than … a systematic image scan.” However, they do not offer even a basic calculation of the reduction in cost.
%%
%%Figure fonts are very small in general. As a rule-of-thumb, the fonts used in figures should roughly be the same size as the main text fonts.
%%
%%What about inhibition of return? Does your model implement it? What prevents your model to jump to a previously foveated position?
%%
%%In Figure 7, what do the blue and red colors mean? There could be several ways to interpret these plots. Please improve the plots and the caption for clarity. Also, the cases B and C are mixed up.
%%
%%The authors mention “sub-linear (logarithmic) visual search”. As a reader I expected more than just a mention. How will this be achieved? This claim is not substantiated by any means.
%%
%%Many references are missing some fields. For example, check refs [6], [8], [25] and [26]. All refs should be checked for formatting and missing fields.
%%
%%The foveal processing module does a 10-way classification. But what about a no-digit example? Does the model predict “no-digit” or “background”? Is there a need for such a decision? Please discuss.
%%
%%Finally, there are lots of typos (which can be easily caught by automatic spell-checkers) and grammatical errors. A non-comprehensive list follows.
%%In the abstract, in the second occurence of “Where”, the letter “W” has an accent on it.
%%Last sentence of the “Author summary” is grammatically not correct.
%%Line 62: “provide” -> provides
%%Line 77: “relies a non-homogeneous” -> check for correctness
%%Line 127: what does “both” refer to?
%%Line 145: “implements” -> implement
%%Line 172: “need to that it’s” ???
%%Line 180: “how typically looks the” ???
%%Figure 2 caption: “wether”
%%Line 279: “position is draw a” ???
%%Line 308: “here the known” ???
%%Line 310: “made of a 3 convolution”
%%Line 346: “ouput”
%%Line 348: “For to reduce” ???
%%Line 363: “These position,”
%%Line 365: “model” -> models
%%Line 383: opening quotation mark should be corrected. Similar typos can be found in the caption of Figure 4.
%%Line 416: “1 hours”
%%Line 435: please check for correctness of grammar
%%Line 446: “of of”
%%Line 491: there should be a period at the end of the sentence
%%Line 560: “lots of model”
%%
%%--------------------
% #######################################################################################################################
% Template for PLoS
% Version 3.5 March 2018
%
% % % % % % % % % % % % % % % % % % % % % %
%
% -- IMPORTANT NOTE
%
% This template contains comments intended
% to minimize problems and delays during our production
% process. Please follow the template instructions
% whenever possible.
%
% % % % % % % % % % % % % % % % % % % % % % %
%
% Once your paper is accepted for publication,
% PLEASE REMOVE ALL TRACKED CHANGES in this file
% and leave only the final text of your manuscript.
% PLOS recommends the use of latexdiff to track changes during review, as this will help to maintain a clean tex file.
% Visit https://www.ctan.org/pkg/latexdiff?lang=en for info or contact us at latex@plos.org.
%
%
% There are no restrictions on package use within the LaTeX files except that
% no packages listed in the template may be deleted.
%
% Please do not include colors or graphics in the text.
%
% The manuscript LaTeX source should be contained within a single file (do not use \input, \externaldocument, or similar commands).
%
% % % % % % % % % % % % % % % % % % % % % % %
%
% -- FIGURES AND TABLES
%
% Please include tables/figure captions directly after the paragraph where they are first cited in the text.
%
% DO NOT INCLUDE GRAPHICS IN YOUR MANUSCRIPT
% - Figures should be uploaded separately from your manuscript file.
% - Figures generated using LaTeX should be extracted and removed from the PDF before submission.
% - Figures containing multiple panels/subfigures must be combined into one image file before submission.
% For figure citations, please use "Fig" instead of "Figure".
% See http://journals.plos.org/plosone/s/figures for PLOS figure guidelines.
%
% Tables should be cell-based and may not contain:
% - spacing/line breaks within cells to alter layout or alignment
% - do not nest tabular environments (no tabular environments within tabular environments)
% - no graphics or colored text (cell background color/shading OK)
% See http://journals.plos.org/plosone/s/tables for table guidelines.
%
% For tables that exceed the width of the text column, use the adjustwidth environment as illustrated in the example table in text below.
%
% % % % % % % % % % % % % % % % % % % % % % % %
%
% -- EQUATIONS, MATH SYMBOLS, SUBSCRIPTS, AND SUPERSCRIPTS
%
% IMPORTANT
% Below are a few tips to help format your equations and other special characters according to our specifications. For more tips to help reduce the possibility of formatting errors during conversion, please see our LaTeX guidelines at http://journals.plos.org/plosone/s/latex
%
% For inline equations, please be sure to include all portions of an equation in the math environment.  For example, x$^2$ is incorrect; this should be formatted as $x^2$ (or $\mathrm{x}^2$ if the romanized font is desired).
%
% Do not include text that is not math in the math environment. For example, CO2 should be written as CO\textsubscript{2} instead of CO$_2$.
%
% Please add line breaks to long display equations when possible in order to fit size of the column.
%
% For inline equations, please do not include punctuation (commas, etc) within the math environment unless this is part of the equation.
%
% When adding superscript or subscripts outside of brackets/braces, please group using {}.  For example, change "[U(D,E,\gamma)]^2" to "{[U(D,E,\gamma)]}^2".
%
% Do not use \cal for caligraphic font.  Instead, use \mathcal{}
%
% % % % % % % % % % % % % % % % % % % % % % % %
%
% Please contact latex@plos.org with any questions.
%
% % % % % % % % % % % % % % % % % % % % % % % %

\documentclass[10pt,a4paper]{article}
% \usepackage[top=0.85in,left=2.75in,footskip=0.75in]{geometry}
\usepackage[top=0.85in,left=1.5in,footskip=0.75in]{geometry}

% amsmath and amssymb packages, useful for mathematical formulas and symbols
\usepackage{amsmath,amssymb}

% Use adjustwidth environment to exceed column width (see example table in text)
\usepackage{changepage}

% Use Unicode characters when possible
\usepackage[utf8x]{inputenc}

% textcomp package and marvosym package for additional characters
\usepackage{textcomp,marvosym}

% cite package, to clean up citations in the main text. Do not remove.
\usepackage{cite}

% Use nameref to cite supporting information files (see Supporting Information section for more info)
\usepackage{nameref,hyperref}

% line numbers
\usepackage[right]{lineno}

% ligatures disabled
\usepackage{microtype}
\DisableLigatures[f]{encoding = *, family = * }

% color can be used to apply background shading to table cells only
\usepackage[table]{xcolor}

% array package and thick rules for tables
\usepackage{array}

\usepackage{csquotes}
\usepackage{bm}

%
% % create "+" rule type for thick vertical lines
% \newcolumntype{+}{!{\vrule width 2pt}}
%
% % create \thickcline for thick horizontal lines of variable length
% \newlength\savedwidth
% \newcommand\thickcline[1]{%
%   \noalign{\global\savedwidth\arrayrulewidth\global\arrayrulewidth 2pt}%
%   \cline{#1}%
%   \noalign{\vskip\arrayrulewidth}%
%   \noalign{\global\arrayrulewidth\savedwidth}%
% }
%
% % \thickhline command for thick horizontal lines that span the table
% \newcommand\thickhline{\noalign{\global\savedwidth\arrayrulewidth\global\arrayrulewidth 2pt}%
% \hline
% \noalign{\global\arrayrulewidth\savedwidth}}
%
%
% % Remove comment for double spacing
% %\usepackage{setspace}
% %\doublespacing
%
% % Text layout
% \raggedright
% \setlength{\parindent}{0.5cm}
% \textwidth 5.25in
% \textheight 8.75in

% Bold the 'Figure #' in the caption and separate it from the title/caption with a period
% Captions will be left justified
\usepackage[aboveskip=1pt,labelfont=bf,labelsep=period,%justification=raggedright,
singlelinecheck=off]{caption}
\renewcommand{\figurename}{Fig}

% Use the PLoS provided BiBTeX style
%\bibliographystyle{plos2015}

% Remove brackets from numbering in List of References
\makeatletter
\renewcommand{\@biblabel}[1]{\quad#1.}
\makeatother

% Header and Footer with logo
\usepackage{lastpage,fancyhdr,graphicx}
\usepackage{epstopdf}
%\pagestyle{myheadings}
\pagestyle{fancy}
\fancyhf{}
%\setlength{\headheight}{27.023pt}
%\lhead{\includegraphics[width=2.0in]{PLOS-submission.eps}}
\rfoot{\thepage/\pageref{LastPage}}
\renewcommand{\headrulewidth}{0pt}
% \renewcommand{\footrule}{\hrule height 2pt \vspace*{2mm}}
% \fancyheadoffset[L]{2.25in}
% \fancyfootoffset[L]{2.25in}
\lfoot{\today}

\usepackage{siunitx}
%\renewcommand{\cite}{\citep}%
\newcommand{\ms}{\si{\milli\second}}%
\newcommand{\m}{\si{\meter}}%
\newcommand{\s}{\si{\second}}%

\newcommand{\FIX}{\texttt{FIX}}%
\newcommand{\DIS}{\texttt{DIS}}%
\newcommand{\SAC}{\texttt{SAC}}%
\newcommand{\ANS}{\texttt{ANS}}%
\newcommand{\A}{\textbf{(A)~}}%
\newcommand{\B}{\textbf{(B)~}}%
\newcommand{\C}{\textbf{(C)~}}%
\newcommand{\D}{\textbf{(D)~}}%
\newcommand{\E}{\textbf{(E)~}}%
\newcommand{\F}{\textbf{(F)~}}%


\newcommand{\lorem}{{\bf LOREM}}
\newcommand{\ipsum}{{\bf IPSUM}}


%% END MACROS SECTION


\begin{document}
% \vspace{0.2in}
% !TEX root = DauceAlbigesPerrinet2020.tex
%!TeX TS-program = pdflatex
%!TeX encoding = UTF-8 Unicode
%!TeX spellcheck = en-US
%!BIB TS-program = bibtex
% -*- coding: UTF-8; -*-
% vim: set fenc=utf-8
%: %%%%%%%%%%%%%%%%%%%%%%%%%%%%%%%%%%%%%%%%%%%%%%%%%%%%%%%%%%%%%%%%%%%%
%: notes
%  journals = https://en.wikipedia.org/wiki/List_of_academic_journals_by_preprint_policy
%
%: METADATA
%: %%%%%%%%%%%%%%%%%%%%%%%%%%%%%%%%%%%%%%%%%%%%%%%%%%%%%%%%%%%%%%%%%%%%
\newcommand{\AuthorPA}{Pierre Albiges}
\newcommand{\AuthorED}{Emmanuel Dauc\'e}%
\newcommand{\AuthorLP}{Laurent Perrinet}%
% \newcommand{\AddressLP}{Institut de Neurosciences de la Timone, CNRS/Aix-Marseille Universit\'e, France}%
% \newcommand{\AddressED}{Institut de Neurosciences des Systèmes, Inserm/Aix-Marseille Universit\'e, France}%
\newcommand{\Address}{Institut de Neurosciences de la Timone, CNRS/Aix-Marseille Universit\'e, France}%
\newcommand{\WebsiteLP}{https://laurentperrinet.github.io/}%
\newcommand{\EmailLP}{Laurent.Perrinet@univ-amu.fr}%
\newcommand{\orcidLP}{0000-0002-9536-010X}%
\newcommand{\orcidED}{0000-0001-6596-8168}%
\newcommand{\Keywords}{Object detection \and Active Inference \and Visual search \and Visuomotor control \and Deep Learning}
\newcommand{\Title}{
%TODO:  (à changer / travailler / masser )
%%Tentative titles by order of preference :
%Learning Where to Look Next for What to See :\\ A Foveated Visual Search Model
%Learning Where to Look Next :\\ A Foveated Visual Search Model%Learning where to see before looking at what it is
%Seeing where before looking what
%Learning where to look: a foveated visuomotor control model
%Looking where it is worth looking
%Learning where to see
%Learning where to look for a target before seeing it
% Learning where it is worth looking at
%Where is My MNIST?
%Where move the eye next? Efficient visual search with foveal vision
%A model of efficient visual search with foveal retina
%Training
%Learning efficient accuracy-seeking action selection in foveated vision % implements visual search
%Training a dual-pathway model for action selection in foveated vision
%Training action selection in foveated vision using a dual-pathway model
%Training saccade selection in foveated vision using a dual-pathway model
%Training saccade selection using a dual-pathway model of foveated vision
% a dual-pathway model of saccade selection using of foveated vision
% a dual-pathway model of saccade selection for foveal visual processing
%A dual-pathway model for foveal visual processing implements saccade selection
%A foveal dual-pathway model implements saccade selection in visual processing
%A foveal-peripheral visual processing model implements saccade selection
%visual processing  using a foveal-peripheral model implements efficient saccade selection
A dual foveal-peripheral visual processing model implements efficient saccade selection
}
\newcommand{\Acknowledgments}{ TODO:  FRM ....... RIck + Karl + Laurent Madelain  - }
\newcommand{\Abstract}{
Visual search involves a dual task of localizing and categorizing an object in the visual field of view. We develop a visuomotor model that implements visual search as a focal accuracy-seeking policy, with the target position and category considered as independently drawn from a common generative process. This independence allows to divide the visual processing in two pathways that respectively infer what to see and where to look, consistently with the anatomical ventral `What'' versus dorsal ``Where'' separation. We use this dual principle to train a deep neural network architecture with the foveal accuracy used as a monitoring signal for action selection. This allows in particular to interpret the ``Where'' network as a retinotopic action selection pathway, that drives the fovea toward the target in order to increase the central recognition accuracy. After training, the comparison of both networks accuracies amounts either to select a saccade or to keep the eye focused at the center, so as to identify the target. We test this on a simple task of finding digits in a large, cluttered image. A biomimetic log-polar treatment of the visual information implements the strong compression rate performed at the sensor level by retinotopic encoding, and is preserved up to the action selection level. Simulation results demonstrate that it is possible to learn this dual network. After training, this dual approach is shown to provide ways to implement visual search in a sub-linear fashion, in contrast with mainstream computer vision.
% {\color{green} Attention à l'usage des temps : présent, passé, futur...}. -> present partout
%
%In computer vision, the visual search task consists in extracting a scarce and specific visual information (the ``target'') from a large and crowded visual display. This task is usually implemented by scanning the different possible target identities at all possible spatial positions, hence with strong computational load. The human visual system employs a different strategy, combining a foveated sensor with the capacity to rapidly move the center of fixation using saccades. Saccade-based visual exploration can be idealized as an inference process, assuming that the target position and category are independently drawn from a common generative process. Knowing that process, visual processing is then separated in two specialized pathways, the ``where'' pathway mainly conveying information about target position in peripheral space, and the ``what'' pathway mainly conveying information about the category of the target. We consider here a dual neural network architecture learning independently where to look and then at what to see. This allows in particular to infer target position in retinotopic coordinates, independently to its category. This framework was tested on a simple task of finding digits in a large, cluttered image. Simulation results demonstrate the benefit of specifically learning where to look before actually knowing the target category. The approach is also energy-efficient as it includes the strong compression rate performed at the sensor level, by retina and V1 encoding, which is preserved up to the action selection level, highlighting the advantages of bio-mimetic strategies with regards to traditional computer vision when computing resources are at stake.
%
% TODO : include whenever we do more than one saccade...
% Without a saccade, the accuracy drops to the baseline at half the width of the target from the center of fixation, while actuating a saccade is beneficial in up to 3 times its size, allowing a much wider covering of the image. The ratio between the marginal accuracies shows that this model is computationally an order of magnitude more efficient than that of a classical brute-force framework. Until the foveal classifier is confident, the system should thus perform saccades to the most likely target position. The different accuracy predictions, such as the ones done in the ``what'' and the ``where'' pathway, may also explain more elaborate decision making, such as the inhibition of return.
%This provides evidence of the importance of identifying ``putative interesting targets'' first and we highlight some possible extensions of our model both in computer vision and modeling.
% TODO: We compared the results of this model with classical psychophysical results in visual search
%This generic visual search problem is of broad interest to machine learning, computer vision and robotics, but also to neuroscience, as it speaks to the mechanisms underlying foveation and more generally to low-level attention mechanisms. From a computer vision perspective, the problem is generally addressed by processing the different hypothesis (categories) at all possible spatial configuration through dedicated parallel hardware. The human visual system, however, seems to employ a different strategy, through a combination of a foveated sensor with the capacity of rapidly moving the center of fixation using saccades.
%Visual processing is done through fast and specialized pathways, one of which mainly conveying information about target position and speed in the peripheral space (the "where" pathway), the other mainly conveying  information about the identity of the target (the "what" pathway). The combination of the two pathways is expected to provide most of the useful knowledge about the external visual scene. Still, it is unknown why such a separation exists.
}
\newcommand{\Precis}{
Separating visual processing into a What and a Where pathways provides a strategy to model visual search. We developed a deep-learning based computational model in which the comparison of predicted accuracies from both pathways allows for efficient saccade selection.
}
\newcommand{\AuthorSummary}{
The visual search task consists in extracting a scarce and specific visual information (the ``target'') from a large and cluttered visual display. In computer vision, this task is usually implemented by scanning all different possible target identities in parallel at all possible spatial positions, hence with strong computational load. The human visual system employs a different strategy, combining a foveated sensor with the capacity to rapidly move the center of fixation using saccades. Then, visual processing is separated in two specialized pathways, the ``where'' pathway mainly conveying information about target position in peripheral space (independently of its category), and the ``what'' pathway mainly conveying information about the category of the target (independently of its position). This object recognition pathway is shown here to have an essential role, providing an ``accuracy drive'' that serves to guide the eye toward peripheral objects in order to increase the peripheral accuracy, much like in the ``actor/critic'' framework. Put together, all those principles are shown to provide ways toward both adaptive and resource-efficient visual processing systems.
}
%%%%%%%%%%%%%%%%%%%%%%%%%%%%%%%%%%%%%%%%%%%
%\documentclass[10pt,a4paper]{llncs}
%\usepackage[T1]{fontenc}
%\usepackage[utf8]{inputenc}
%
%\usepackage{graphicx}
%\usepackage{geometry}
%
%\usepackage{amsmath}
%\usepackage{amssymb}
%\usepackage{a4}
%
%\usepackage{csquotes}
%\usepackage{bm}
%
%\usepackage{graphicx}
%\DeclareGraphicsExtensions{.pdf}%,.png,.jpg}
%%\graphicspath{{.}}%
%
%\usepackage{color}
%
%%opening
%\title{
%\Title
%%\thanks{\Acknowledgments }
%}
%\author{\AuthorED \inst{1}\orcidID{\orcidED} \and \AuthorPA \inst{1,2} \and \AuthorLP \inst{2}\orcidID{\orcidLP }  }
%\institute{\AddressED
%\and \AddressLP
%%\if 1\ICANN \else
%%\\  \email{\EmailLP } \url{https://laurentperrinet.github.io/} \fi
%}
%\date{}
%%============ bibliography ===================
%%\usepackage[numbers,comma,sort&compress,round]{natbib} %
%\usepackage[
%%style=alphabetic-verb,
%style=authoryear-comp,
%%style=apa,
%%maxcitenames=2,
%%maxnames = 2,
%%giveninits=true,
%%uniquename=init,
%%sorting=none,
%doi=true,
%url=false,
%isbn=false,
%eprint=true,
%texencoding=utf8,
%bibencoding=utf8,
%autocite=superscript,
%backend=bibtex,
%%articletitle=false
%]{biblatex}%
%%\addbibresource{Bibliography.bib}%
%\bibliography{Bibliography.bib} % the ref.bib file
%\newcommand{\citep}[1]{\parencite{#1}}
%\newcommand{\citet}[1]{\textcite{#1}}
%%%%%%%%%%%%%%%%%%%%%%%%%%%%%%%
%\usepackage{siunitx}
%%\renewcommand{\cite}{\citep}%
%\newcommand{\ms}{\si{\milli\second}}%
%\newcommand{\m}{\si{\meter}}%
%\newcommand{\s}{\si{\second}}%
%
%\newcommand{\FIX}{\texttt{FIX}}%
%\newcommand{\DIS}{\texttt{DIS}}%
%\newcommand{\SAC}{\texttt{SAC}}%
%\newcommand{\ANS}{\texttt{ANS}}%
%\newcommand{\A}{\textbf{(A)~}}%
%\newcommand{\B}{\textbf{(B)~}}%
%\newcommand{\C}{\textbf{(C)~}}%
%\newcommand{\D}{\textbf{(D)~}}%
%\newcommand{\E}{\textbf{(E)~}}%
%\newcommand{\F}{\textbf{(F)~}}%
%
%%	% \usepackage{times}
%%	%\inner 0.5in
%%	% \oddsidemargin -0.5in		% margin, in addition to 1" standard
%%	% \textwidth 17cm		% 8.5" - 2*(1+\oddsidemargin)
%%
%%	% \topmargin -1in		% in addition to 1.5" standard margin
%%	% \textheight 10.69in 		% 11 - ( 1.5 + \topmargin + <bottom-margin> )
%%
%%	% \columnsep 0.25in
%%	%
%%	\parindent 0pt
%%	\parskip 0pt
%%
%%\setlength{\parskip}{0pt}
%%\setlength{\parsep}{0pt}
%%\setlength{\headsep}{0pt}
%%\setlength{\topskip}{0pt}
%%\setlength{\topmargin}{0pt}
%%\setlength{\topsep}{0pt}
%%\setlength{\partopsep}{0pt}
%%
%%%\usepackage[compact]{titlesec}
%%%\titlespacing{\section}{0pt}{*0}{*0}
%%%\titlespacing{\subsection}{0pt}{*0}{*0}
%%%\titlespacing{\subsubsection}{0pt}{*0}{*0}
%%	% \usepackage{titlesec}
%%    %
%%	% \titlespacing\section{0pt}{12pt plus 4pt minus 2pt}{0pt plus 2pt minus 2pt}
%%	%\titlespacing\subsection{0pt}{12pt plus 4pt minus 2pt}{0pt plus 2pt minus 2pt}
%%	%\titlespacing\subsubsection{0pt}{12pt plus 4pt minus 2pt}{0pt plus 2pt minus 2pt}
%%
%%%\flushbottom \sloppy
%%%\pagestyle{empty} % No page numbers
%%
%%
%%\renewcommand{\paragraph}{\emph}%
%
%
%\begin{document}
%
%\maketitle
%
%\begin{abstract}
%\Abstract
%
%\keywords{\Keywords}
%
%
%\end{abstract}
%%
%%\newpage
%
%% !TEX root = paper.tex
% !TEX encoding = UTF-8 Unicode
% -*- coding: UTF-8; -*-
% vim: set fenc=utf-8
% !TEX spellcheck = en-US
\section{Introduction}
\label{sec:intro}
\paragraph{Problem statement.}
%------------------------------%
%: see Figure~\ref{fig:intro}
\begin{figure}[t!]%[b!]%%[p!]
	\centering{	\includegraphics[width=\linewidth]{fig_intro}} %
	\caption{%
		{\bf Problem setting}: In generic, ecological settings, the visual system faces a tricky problem when searching for one target (from a class of targets) in a cluttered environment. It is synthesized in the following experiment: %
		\A After a fixation period \FIX\ of $200~\ms$, an observer is presented with a luminous display \DIS\ showing a single target from a known class (here digits) and at a random position. The display is presented for a short period of $500~\ms$ (light shaded area in B), that is enough to perform at most one saccade on the potential target (\SAC , here successful). Finally, the observer has to identify the digit by a keypress \ANS . %
		\B Prototypical trace of a saccadic eye movement to the target position. In particular, we show the fixation window \FIX\ and the temporal window during which a saccade is possible (green shaded area). %
		\C Simulated reconstruction of the visual information from the (interoceptive) retinotopic map at the onset of the display \DIS\ and after a saccade \SAC , the dashed red box indicating the visual area of the ``what'' pathway. In contrast to an exteroceptive representation (see A), this demonstrates that the position of the target has to be inferred from a degraded (sampled) image. In particular, the configuration of the display is such that by adding clutter and reducing the size of the digit, it may become necessary to perform a saccade to be able to identify the digit. The computational pathway mediating the action has to infer the location of the target \emph{before seeing it}, that is, before being able to actually identify the target's category from a central fixation. %
		\label{fig:intro}}%
\end{figure}%
%%------------------------------%

The promise of artificial vision to identify objects in natural images is ever increasing. Image processing algorithms recently outreached the performance of human observers in specific image categorization tasks~\citep{He15}. Initially trained on energy greedy, high performance computers, they are now designed to work on more common hardware such as desktop computers with dedicated GPU hardware~\citep{Sandler18}. However, these algorithms are still far from human performances, even for simple tasks. Take for instance the case of an encounter with a friend in a crowded café. To catch the moment at which she arrives, you need to visually search for her face despite the sensory clutter in the visual field. To do so, you need to scan relevant parts of the visual scene with your gaze. Doing a saccade at these locations will allow you to recognize your friend. The main difficulty of this task is to learn to categorize this particular object class given all possible spatial configurations and respective geometrical visual transformations. 

This visual search experience can be formalized and simplified in a way reminiscent to classical psychophysical experiments: an observer is asked to classify digits (for instance as taken from the MNIST database) as they are shown on a computer display. However, these digits can be placed at random positions on the display, and visual clutter is added as a background to the image (see Figure~\ref{fig:intro}-A). This opens the possibility that the position of the object may be detected in the clutter without being identified in the first place (see Figure~\ref{fig:intro}-C). This defines more precisely our problem: how do we localize an object in a large image while knowing \emph{a priori} its category but not its identity? This generic visual search problem is of broad interest in machine learning, computer vision and robotics, but also in neuroscience, as it speaks to the mechanisms underlying foveation and more generally to low-level attention mechanisms.

Inherent to this problem is the combinatorial explosion implied by an increasing number of parameters. State-of-the art classification architectures consequently contain many millions parameters with subsequent energy consumption increase while still handling relatively small images. This introduces a trade-off between efficiency and average accuracy, for instance in autonomous driving such that the algorithm is fast enough to detect visual objects in a glance while running on resource-constrained devices like embedded devices. Globally, this performance is still lower than that of humans. Indeed, the human visual system can perform such a feat both rapidly, --~in less than 100 ms~\citep{Kirchner06}~-- and at a low energy cost ($<5~W$). On top of that, it is mostly self-organized, robust to visual transforms or lighting conditions and can learn with a few examples. If many different anatomical features may explain this efficiency, a main difference lies in the fact that its sensor (the retina) combines a non homogeneous sampling of the world with the capacity to rapidly change its center of fixation. Indeed, on the one hand, the retina is composed of two separate systems: a central, high definition fovea (a disk of about 6 degrees of diameter in visual angle around the center of gaze) and a large, lower definition peripheral area. On the other hand, the retina is attached on the back of the eye which is capable of low latency, high speed eye movements. In particular, saccades allow for efficient changes of the position of the center of gaze: they take about $200~\ms$ to initiate, last about $200~\ms$ and usually reach a maximum velocity of approx 600 degrees per second. This behavior is prevalent during our lifetime (about a saccade every 2-3 seconds, that is, almost a billion saccade in a lifetime). The interplay of those two features allows human observers to engage in an integrated action perception loop which sequentially scans and analyses the different parts of the image.
%It is one type of active inference~\citep{Friston12} (see below) and we will envision herein how to incorporate it to classical computer vision schemes.
% (1 / 2.5 * 3600 * 24 * 365 * 75 = 946080000.0 ~= .95e9) X (wakeful + REM = .66)
%
\paragraph{State of the art.}

To take advantage of this visuomotor behavior, it is of particular importance to understand both its computational and neurophysiological principles. First, the joint problem of target localization and identification is a classical problem of visual search in computer vision. It is very general and may address apparently simple questions such as ``find the green bottle on the table''. 
When restricted to a mere ``feature search''~\citep{Treisman80}, many solutions are proposed. Notably, recent advances in deep-learning have provided efficient models such as faster-RCNN~\citep{Ren17} or YOLO~\citep{Redmon15}. 
This last implementation is particularly interesting for our sake as it predicts in the image the probability of proposed bounding boxes around visual objects. While rapid, the number of boxes greatly increases with image size and necessitates dedicated hardware. 
In parallel, when limited to a few objects of interest in the image, this strategy amounts to a classical problem in neuroscience, that is, the transformation of a luminous image into a saliency map~\citep{Itti01}, essential to understand and predict saccades, but also to serve as phenomenological models of attention. The saliency approach was recently extended using deep learning to estimate saliency maps over large databases of natural images~\citep{Kummerer16}. While these methods are efficient at predicting the probability of fixation, they miss an essential component in the action perception loop: they operate on the full image while the retina operates on the non-uniform, foveated sampling of visual space (see Figure~\ref{fig:intro}-B). 
Herein, we believe that this fact is an essential factor to reproduce and understand this active vision process.

In contrast to phenomenological (or ``bottom-up'') approaches, models of active vision~\citep{Najemnik05,Butko2010infomax,Friston12} provide the ground principles of saccadic exploration. In general, they assume the existence of a generative model from which both the target position and category can be inferred through active sampling. This comes from the constraint that the visual sensor is foveated but can generate a saccade. 
Several studies are relevant to our endeavor. First, one can consider optimal strategies to solve the problem of the visual search of a target~\citep{Najemnik05}. In a setting similar to that presented in Figure~\ref{fig:intro}-A, where the target is an oriented edge and the background is defined as pink noise, authors show first that a Bayesian ideal observer comes out with an optimal strategy, and second that human observers are close to that optimal performance. Though well predicting sequences of saccades in a perception action loop, this model is limited by the simplicity of the display (elementary edges added on stationary noise, a finite number of locations on a discrete grid) and by the abstract level of modeling. Despite these (inevitable) simplifications, this study could successfully predict some key characteristics of visual scanning such as the trade-off between memory content and rapidity. Looking more closely at neurophysiology, the study of~\citep{Samonds18} allows to go further in understanding the interplay between saccadic behavior and the statistics of the input. In this study, authors were able to manipulate the size of the saccades by monitoring key properties of the presented (natural) images. For instance, smaller images generate smaller saccades. Interestingly, they also predicted the size of saccades for different species, including mice which lack a foveal region, from the size of visual receptive fields. One key prediction of this study which is relevant for our problem is the fact that saccades seem optimal to \emph{a priori} decorrelate the visual input, that is, to minimize redundancy in the sequence of generated saccades, knowing the statistics of the visual inputs.

A further modeling perspective is provided by~\citep{Friston12}. In this setup, a full description of the visual world is used as a generative process. An agent is completely described by giving the generative model governing the dynamics of its internal beliefs and is interacting with this image by scanning it through a foveated sensor, just as described in Figure~\ref{fig:intro}. Thus, equipping the agent with the ability to actively sample the visual world %enables to explore the idea that actions (saccadic eye movements) are 
allows to interpret saccades as optimal experiments, by which the agent seeks to confirm predictive models of the (hidden) world. One key ingredient to this process is the (internal) representation of counterfactual predictions, that is, the probable consequences of possible hypothesis as they would be realized into actions (here, saccades). Following such an active inference scheme~\citep{Mirza18} numerical simulations reproduce sequential eye movements that fit well with empirical data. %Compared to~\citet{Najemnik05}, 
Saccades %are not the output of a value-based cost function, but 
are here a consequence of an active seek for the agent to minimize the uncertainty about his beliefs, knowing his priors on the generative model of the visual world. 

\paragraph{Outline.}
Stemming from the active vision general principles, our aim is to produce a principled model that may both explain the essential features of human vision and provide ways toward efficient computer implementations. We also aim at reunifying the fragmentation of the many different approaches respective to their fields (Machine learning, neuroscience, robotics), and envisage an integrated computational model of foveated active vision. It is known that inverting a generative model over a large (one-step ahead) hypothesis space of all possible saccades is computationally-intensive. % (think for instance of face category as a very large categorical space over a large visual transformation space) with no obvious neurophysiological counterpart. (see Figure~\ref{fig:intro}-C)
%Although we similarly include a generative process of the visual world,
Herein, we hypothesize that complex combinatorial inferences can be replaced by separate pathways, i.e. the spatial (``where'') and categorical (``what'') pathways, whose knowledge is combined to infer optimal eye displacements and subsequent identification of the target. 
%as conatining {\bf (containing??)} images of a handwritten random digit (drawn from the MNIST database) at a random position and embedded in a cluttered noise . 
We will thus define an agent equipped with a foveated sensor and with the ability to actively scan the visual image, %. % as defined by a generative (internal) modelWe will use this constraint as an asset 
%to which also contributes to minimizing the overall computational cost of finding a target. 
%Taking such priors, we 
learn an optimal behavior startegy and explore its key properties.

This paper is organized as follows: After this introduction, we define the principles underlying accuracy-based saccadic control in section~\ref{sec:principles}. We first define notations, variables and equations for the generative process governing the experiment and the generative model for the active vision agent. In particular, we derive our method to simplify the learning of an optimal agent given these definitions. In section \ref{sec:implementation}, implementation details are given, providing ways to reproduce our results. In section~\ref{sec:results}, preliminary results of numerical simulations of the agent are presented, demonstrating the applicability of this framework to different task complexity levels. This allows us to derive some limits of the agent and, as in~\citep{Najemnik05}, we draw some analogies with biologically observed eye movements. Finally, in section~\ref{sec:discussion}, we summarize these results in comparison with other similar schemes. We conclude by showing the relative advantages of using this active inference approach.

%%
%%\newpage
%
%% !TEX root = paper.tex
% !TEX encoding = UTF-8 Unicode
% -*- coding: UTF-8; -*-
% vim: set fenc=utf-8
% !TEX spellcheck = en-US
\section{Methods}
In this study, the visual scene is made of an unknown target at a random position and a noisy background (see Figure~\ref{fig:intro}). An agent controls a focal visual sensor that can move over the visual scene through saccades. In the implementation of such networks, we will follow the simplifying assumption that there is a separation between the inferences of  position and category in two respective pathways, namely the ``What'' and the ``Where'' pathways. The ``What'' pathway will be given from the literature and to test the validity of our hypothesis, it is necessary to find at least one function implementing the ``where'' network and that would be able to find the position of an object knowing only the degraded retinal image. Here, we describe the methods that we will follow to find that function, from the generative models (first external and then internal) to the actual implementation of that ``where'' pathway knowing a fixed ``what'' pathway. %
%There are however many shortcuts allowing to render the calculation amenable. This includes (i) sparse encoding, (ii) approximate inference through model separation, and (iii) sampling-based metric training. 
%------------------------------%
%: see Figure~\ref{fig:methods}
\begin{figure}[t!]%%[p!]
\centering{\includegraphics[width=\linewidth]{fig_intro}}%{fig_methods}}
\caption{
{\bf Methods for simulating active vision}:
\A We first define the model which generates images. It is composed of different random processes: one choosing a sample image from the MNIST database (of size $28\times 28$) and placing it at a random position within the circular mask on the $128\times 128$ display. Then, this image is rectified and multiplied by a contrast factor and finally embedded in a natural-like noise with characteristics its contrast, mean spatial frequency and bandwidth~\citep{Sanz12}. %
\B The full-sized images are transformed into a retinal image which will be fed to the ``where'' pathway. This is implemented by a bank of filters whose centers are centered of a log-polar grid and whose radius increases proportionally to eccentricity. Crucially, a similar transform is used to compute the accuracy of each hypothetical saccade, as represented by the collicular map. %
\C The ``where'' pathway is implemented by a three-layered neural network consisting of the retinal input, two hidden layers with $1000$ units each and a collicular output. Each unit is associated with a ReLU non-linearity. To learn to associate the output of the network with the ground truth, supervised training is performed using back-propagation with a binary cross entropy loss which measure the distance between both distributions. The network learns in about 20 epochs as shown by the decrease of the loss function. Overlaid is the associated accuracy of the full active agent. This is computed by classifying the foveal image using the ``what'' pathway, after centering the gaze using the result of the ``where pathway''. This shows a gradual increase in accuracy from the baseline ($10\%$) to approximately an average of $X80.0X\%$. %
}%
\end{figure}%
%%------------------------------%

\subsection{Exteroceptive Generative model}
It is first necessary to quantitatively define the generative model for input display images as shown first in Figure~\ref{fig:intro}-A (\DIS ) and implemented in see Figure~\ref{fig:methods}-A. 

\paragraph{Targets.} Following a common hypothesis regarding active vision, visual scenes will consist of a single visual object of interest. We will use the MNIST database of handwritten digits introduced by~\citep{Lecun1998} as classification solutions (``what'' pathway) abound for this class of targets and that we are here focused on the problem of localization (``where'' pathway). Samples are drawn from a database of $60000$ grayscale $28\times 28$ pixels images. 
\paragraph{Full-scale images.} For each sample, we may now draw a random position in a full-scale image of $128\times 128$. To enforce isotropic saccades, we define a centered circular mask covering the image (of radius $64$ pixels) and the position is such that the embedded sample fits entirely into that circular mask.
\paragraph{Background noise setting. } To provide with a realistic background noise, we generated synthetic textures~\citep{Sanz12} using a third random process. These textured images are of the same size of the full-image. These static images are designed to fit well with the statistics of natural images. We chose an isotropic setting where textures are characterized by solely two parameters. One controls the median spatial frequency $sf_0$ of this noise, while the other controls the bandwidth around this central spatial frequency. Finally, these can be considered as band-pass filtered images of random white noise. Finally, these images are rectified to have a normalized contrast.
\paragraph{Adding signal and noise. } Finally, both the noise and the target image are merged into a single image. We have used two different strategies. In a first strategy emulating a transparent association, we computed the average luminance at each pixel, while in a second strategy emulating an opaque association, we choose for each pixel the maximal value.
The quantitative difference were tested in simulations, but proved to have a marginal importance.
\subsection{Interoceptive generative model}


\paragraph{Foveal vision and the ``what'' pathway}
First, foveal vision is defined as the $28\times 28$ pixels image centered at the point of fixation (see dashed red box in Figure~\ref{fig:intro}-C). This image is then directly passed to the agent's visual categorical pathway (the ``What'' pathway). This is realized by the known ``LeNet'' classifier~\citep{Lecun1998}, that processes the $28 \times 28$ central pixels to identify the target category. Such a network is provided by the pyTorch library~\citep{Paszke17}, and consists of a 3-layered Convolutional Neural Network. It is trained over the (centered) MNIST database after approx $20$ training epochs. Input images are rectified (with a mean and standard deviation of respectively $0.1307$ and $0.3081$). The network outputs a vector representing the probability of detecting each of the $10$ digits. When taking the maximum probability, it achieves an average $98.7\%$ accuracy on a test dataset~\citep{Lecun1998}. % 

\paragraph{Retinal transform: Peripheral vision and log Polar encoding}
% >>> Laurent is here <<<
First, both the visual features and the expected target position may to be expressed in log-polar coordinates. On the primary visual side, local visual features (first and second order orientation filters) are radially organized around the center of fixation, with small and tightened receptive fields at the center and more large and scarce receptive fields at the periphery. The issued observation vector $\boldsymbol{x}$ compresses the original image by about 90\%, with high spatial frequencies preserved at the center and only low spatial frequencies conserved at the periphery.

The full-sized images are transformed into a retinal image which will be fed to the ``where'' pathway. This is implemented by a bank of filters whose centers are centered of a log-polar grid and whose radius increases proportionally to eccentricity. Crucially, a similar transform is used to compute the accuracy of each hypothetical saccade, as represented by the collicular map. %



\paragraph{Collicular representation: Metric training}

 In particular, we could also evaluate after this training phase the accuracy map of the classifier knowing a translational shift imposed to the input image. 


%The target accuracy map is also organized radially in a log-polar fashion, making the target position estimate more precise at the center and fuzzier at the periphery. This modeling choice is reminiscent of the approximate log-polar organization of the superior colliculus (SC) motor map {\bf[TODO:REF]}.
%This retinotopic organization is preserved along the visuo-motor pathway as expected from observations {\bf[TODO:REF]}.


%Though the effect of action is too complex to be inferred from a generative model, we assume here that it is trained by sampling, i.e. by "trial and error".

%The target accuracy map is also organized radially in a log-polar fashion, making the target position estimate more precise at the center and fuzzier at the periphery. This modeling choice is reminiscent of the approximate log-polar organization of the superior colliculus (SC) motor map {\bf[TODO:REF]}.
%This retinotopic organization is preserved along the visuo-motor pathway as expected from observations {\bf[TODO:REF]}.

 This central classifier displays a high accuracy at the center, and a fast decreasing accuracy with target eccentricity, as shown in figure \ref{fig:results}-D. In contrast, the visual orientation pathway (the ``Where'' pathway) takes the full visual field into account in order to tell whether a target is present at the different peripheral locations, in order to monitor future saccades.



%Though the effect of action is too complex to be inferred from a generative model, we assume here that it is trained by sampling, i.e. by "trial and error".


%<<<<<<< HEAD
%A second simplifying assumption is that the putative effect of a saccade should be condensed in a single number, the \emph{accuracy}, that is a statistics over the (scene understanding) benefit obtained from past saccades in the same context, independently of the identity of the visual objects. In detail, the primary visual information should be transformed so as to predict how accurate the categorical classifier will be after the saccade is carried out~\citep{Dauce18}. The set of all possible saccade predictions should form an \emph{accuracy map}.
%An accuracy map abstracts here a full sequence of operations, including (i) an initial visual examination, followed by (ii) a decision, (iii) a saccade realization and a (iv) second visual examination that should finally (v) determine the category of the target. 
%It should be mostly organized radially, preserving the initial retinotopic organization, with high predicted accuracies reflecting a high probability of target presence at given locations. 
%Such a \emph{predictive accuracy map} is assumed to be the cornerstone of a realistic saccade-based vision system, with action selection (motor map) overlaying the accuracy map through a winner-takes-all mechanism (as thought to be done in the superior colliculus). Of course, each different initial visual field comes with a different accuracy map (essentially conveying information about the target retinotopic position).
%Our main argument is that such an accuracy map is trainable in a rather straightforward way, through trials and errors, by actuating saccades after processing the visual input, and taking the final classification success or failure as a teaching signal. 
%\fi
%=======
%<<<<<<< HEAD

Active inference assumes a hidden emitter $e$, which is known indirectly through its effects on the sensor, that obey to a generative process : $x\sim p(X|e)$. The real emitter state $e$ being hidden, a parametric model $\theta$ is assumed to allow estimate the cause of the current visual field through model inversion thanks to Bayes formula, in short:
$$p(E|x) \propto p(x|E;\theta)$$
with $x$ the visual field in our case. Assume now that the cause $e$ of the visual field splits in two (independent) components, namely $e = (u,y)$ with $u$ the body posture (in our case the gaze orientation) and $y$ the object shape (or object identity). Assume also that a set of motor commands $A = \{..., a, ...\}$ may control the body posture, but not the object's identity, so that $y$ is invariant to $a$.
Then, before taking a decision, the consequence of every saccade should be analyzed  through model inversion \emph{over the future observations}, that is predicting the effect of every $a$ in $A$ to choose the action that may optimizes future inferences. The benefit of each $a$ is quantified through a certain metric (future accuracy, future posterior entropy, future variational free energy, ...), that depend on the current inference $p(U,Y|x)$. Each saccade $a$ is thus expected to provide a new visual sample from a given scene statistics, which may increase the understanding of the scene (here the target position and category). However, estimating the effect of every action over the range of every possible object shapes and body postures is combinatorially hard, even in simple cases such as vision, and thus infeasible in practice. 


%Though the effect of action is too complex to be inferred from a generative model, we assume here that it is trained by sampling, i.e. by "trial and error".

There are however many shortcuts allowing to render the calculation amenable. This includes (i) sparse encoding, (ii) approximate inference through model separation, and (iii) sampling-based metric training. 

\paragraph{Sparse encoding}
First, both the visual features and the expected target position may to be expressed in log-polar coordinates. On the primary visual side, local visual features (first and second order orientation filters) are radially organized around the center of fixation, with small and tightened receptive fields at the center and more large and scarce receptive fields at the periphery. The issued observation vector $\boldsymbol{x}$ compresses the original image by about 90\%, with high spatial frequencies preserved at the center and only low spatial frequencies conserved at the periphery.

\paragraph{Approximate inference}
Second, we %start as in~\citep{Friston12} by a probabilistic formulation, and 
use the fundamental hypothesis outlined in Figure~\ref{fig:intro}: the position of an object is independent from its category.  This allows considering $u$ and $y$ being independently inferred from the current visual field, i.e $p(U,Y|x) = p(U|x) p(Y|x)$. This property is strictly true in our setting and is very generic in vision for simple classes (such as digits) and simple displays (but see~\citep{Vo12} for more complex visual scene grammars). 
From this independence hypothesis, we may separate both inferences (identification vs localization) in two separate pathways with different morphologies (respectively foveal and peripheral). Note that from the retinotopic projection of the visual information, this independence is conditional on action: both pathways should update their beliefs upon decisions made in each respective pathway {\bf (??)}.
%A first simplifying assumption is a separation of the position and category inferences in two separate pathways, namely the ``What'' and the ``Where'' pathways.
The agent visual categorical pathway (the ``What'' pathway) is supposed to be realized by the known ``LeNet'' classifier~\citep{Lecun1998}, that processes the $28 \times 28$ central pixels to identify the target category (see dashed red boxes in  Figure~\ref{fig:intro}-C). This central classifier displays a high accuracy at the center, and a fast decreasing accuracy with target eccentricity, as shown in figure \ref{fig:results}-D. In contrast, the visual orientation pathway (the ``Where'' pathway) takes the full visual field into account in order to tell whether a target is present at the different peripheral locations, in order to monitor future saccades.

\paragraph{Metric training}
%=======
%>>>>>>> 10eba6746a264a8f6141953d7948057e5093489d
Third, the putative effect of every saccade should be condensed in a single number, the \emph{accuracy}, that is the expected benefit of issuing saccade $a$ %regarding the target identity, both assuming $p(U|\boldsymbol{x})$ and $p(Y|\boldsymbol{x})$ 
from the current observation. Taking $a$ a possible saccade and $\tilde{\boldsymbol{x}}$ the corresponding future visual field, the result of the categorical classifier over $\tilde{\boldsymbol{x}}$ can either be correct (1) or incorrect (0). 
If this experiment is repeated many times over many visual scenes, the probability of correctly classifying the future visual field $\tilde{\boldsymbol{x}}$ after a saccade $a$ forms a probability, i.e. a number between 0 and 1, that reflects the proportion of correct and incorrect classifications.
% when issuing a saccade $a$ after seeing $\boldsymbol{x}$ (the initial visual field). 
It more or less corresponds to inferring the true target identity $\hat{y}$, i.e. $p(\hat{y}|\tilde{\boldsymbol{x}})$, including the update of the eye direction, that is a sample of the ``real'' generative process. Active inference needs either the current identity $y$ or the current eye direction $u$ to be readable from the present view, in order to effectively predict future inferences, through computationally intensive predictions.   
Instead of doing predictions from a generative model, better off is to form a statistics over the (scene understanding) benefit obtained from past saccades in the same context, that is forming an \emph{accuracy map} from the current view. This is the essence of \emph{sampling-based metric prediction}.

In detail, the primary visual field should be transformed so as to predict how accurate the categorical classifier will be after the saccade is carried out~\citep{Dauce18}. %The set of all possible saccade predictions should 
An accuracy map abstracts here a full sequence of operations, including ($i$) an initial visual examination, followed by ($ii$) a decision, ($iii$) a saccade realization and ($iv$) a second visual examination that should finally ($v$) determine the category of the target.
It should be mostly organized radially, preserving the initial retinotopic organization, with high predicted accuracies reflecting a high probability of target presence at given locations.
Such a \emph{predictive accuracy map} is assumed to be the core of a realistic saccade-based vision system, with action selection (motor map) overlaying the accuracy map through a winner-takes-all mechanism (as thought to be done in the superior colliculus). Of course, each different initial visual field comes with a different accuracy map (essentially conveying information about the target retinotopic position).
Our main argument is that such an accuracy map is trainable in a rather straightforward way, through trials and errors, by actuating saccades after processing the visual input, and taking the final classification success or failure as a teaching signal.

\subsection{Implementing the ``where'' pathway}

This fundamental hypothesis was outlined in Figure~\ref{fig:intro}: the position $u$ of an object is independent from its category (identity) $y$. This allows considering $u$ and $y$ being independently inferred from the current visual field, i.e $p(U,Y|x) = p(U|x) p(Y|x)$. This property is strictly true in our setting and is very generic in vision for simple classes (such as digits) and simple displays (but see~\citep{Vo12} for more complex visual scene grammars). 
From this independence hypothesis, we may separate both inferences (identification vs localization) in two separate pathways with different morphologies (respectively foveal and peripheral). Note that from the retinotopic projection of the visual information, this independence is conditional on action: both pathways should update their beliefs upon decisions made in each respective pathway.
To check this hypothesis, we train a multi-layer neural network to predict accuracy maps from a retinotopic log-polar encoding of the visual image. 


\paragraph{Approximate inference}
Active inference assumes a hidden emitter $e$, which is known indirectly through its effects on the sensor, that obey to a generative process : $x\sim p(X|e)$. The real emitter state $e$ being hidden, a parametric model $\theta$ is assumed to allow estimate the cause of the current visual field through model inversion thanks to Bayes formula, in short:
$$p(E|x) \propto p(x|E;\theta)$$
with $x$ the visual field in our case. Assume now that the cause $e$ of the visual field splits in two (independent) components, namely $e = (u,y)$ with $u$ the body posture (in our case the gaze orientation) and $y$ the object shape (or object identity). Assume also that a set of motor commands $A = \{..., a, ...\}$ may control the body posture, but not the object's identity, so that $y$ is invariant to $a$.
Then, before taking a decision, the consequence of every saccade should be analyzed through model inversion \emph{over the future observations}, that is predicting the effect of every $a$ in $A$ to choose the action that may optimizes future inferences. The benefit of each $a$ is quantified through a certain metric (future accuracy, future posterior entropy, future variational free energy, ...), that depend on the current inference $p(U,Y|x)$. Each saccade $a$ is thus expected to provide a new visual sample from a given scene statistics, which may increase the understanding of the scene (here the target position and category). However, estimating the effect of every action over the range of every possible object shapes and body postures is combinatorially hard, even in simple cases such as vision, and thus infeasible in practice. 





\paragraph{Classifier training}
Modern parametric classifiers are composed of many layers (hence the term ``Deep Learning'') that can be trained through gradient descent over arbitrary input and output feature spaces {\bf [REF?]}. The ease of use of those tightly optimized training algorithms allow to quantify the difficulty of a task through training failure or success. Consider the visual features $\boldsymbol{x}$ as the input and a logpolar retinotopic vector $\boldsymbol{a}$ made of $n$ Bernouilli probabilities (success probabilities) as the output. A textured background noise is used at the input so that (i) the target statistics should barely differ from that of the background and (ii) a categorical classifier may not be trainable with the initial view as the input. A training set is made of randomly generated noisy images, with a low-contrast $28\times 28$ MNIST character  {\bf [REF?]} randomly positioned around the center of fixation (at maximum $30$ pixels from the center of fixation). The images are first whitened {\bf [REF?]}, and then transformed through $16384$ oriented log-polar filters radially organized around the center of fixation, with tight receptive fields at the center and large receptive fields at the periphery. The filters are organized in $10$ spatial eccentricity scales (respectively placed at around $2$, $3$, $4.5$, $6.5$, $9$, $13$, $18$, $26$, $36.5$ , and $51.3$ pixels from the center), and $16$ angular directions, allowing them to cover most of the original 128 $\times$ 128 image. Each visual input is accompanied with a corresponding accuracy map also organized radially through $10$ eccentricity scales and $16$ peripheral directions. The training is done in Pytorch {\bf [REF?]}. The parametric neural network is made of two fully connected hidden layers of size $1000$, with ReLu activation and $50 \%$ drop-out on the last layer. The network is trained over $600,000$ saccades on full-images, using the binary cross-entropy loss as the error signal, with a learning rate equal to $10^{-4}$. The training is done in about 3 hours on a laptop.



%%
%%\newpage
%
%% !TEX root = paper.tex
% !TEX encoding = UTF-8 Unicode
% -*- coding: UTF-8; -*-
% vim: set fenc=utf-8
% !TEX spellcheck = en-US
%=================================================================
\section{Results}
\label{sec:results}
%=================================================================
%------------------------------%

\begin{figure}[t!]%%[p!]
	%\flushleft{\bf (A) \hspace{4.2cm} (B) \hspace{2cm} (C) \hspace{4cm} (D)\hspace{6cm}}
	\centering{{\bf a.} \hspace{1.6cm} {\bf b.} \hspace{1.6cm} {\bf c.} \hspace{1.6cm} {\bf d.} \hspace{1.6cm} {\bf e.}}
	\centering{\A\includegraphics[width=.9\linewidth]{CNS-saccade-8.png}}
	\centering{\B\includegraphics[width=.9\linewidth]{CNS-saccade-20.png}}
	\centering{\C\includegraphics[width=.9\linewidth]{CNS-saccade-46.png}}
	\centering{\D\includegraphics[width=.9\linewidth]{CNS-saccade-47.png}}
	\centering{\E\includegraphics[width=.9\linewidth]{CNS-saccade-32.png}}		% \includegraphics[]{../../2019-07-15_CNS/figures/CNS-saccade-8.png} % TRUE
	\caption{
		{\A-- \E Active vision samples after training}. \A-- \B  classification success samples. \C-- \E classification failure samples. Digit contrast set to 0.7.  From left to right :
		{\bf a.} The initial 128$\times$128 visual display, with blue cross giving the center of gaze.  
		The visual input is retinotopically transformed and sent to the multi-layer neural network implementing the ``Where'' pathway. 
		{\bf b.} Magnified reconstruction of the  visual input, as it shows off from the primary visual features through an inverse log-polar transform. 
		{\bf c.-d.} Color-coded radial representation of the output accuracy maps, with dark violet for the lower accuracies, and yellow for the higher accuracies. The network output ('Predicted') is visually compared with the ground truth ('True'). %
		{\bf e.} $28 \times 28$ central snippet as extracted from the visual display after doing a saccade, with label prediction and success flag in the title. 
		\label{fig:saccades}}%
\end{figure}%

%%------------------------------%
%=================================================================
\subsection{Inferring the target label from active visual search}
%=================================================================
After training, the ``Where'' pathway is now capable to predict an accuracy map, whose maximal argument drives the eye toward a new viewpoint. There, a central snippet is extracted, that is processed through the ``What'' pathway, allowing to predict the digit's label. Examples of this simple open loop sequence are presented in figure \ref{fig:saccades}, when the digits contrast parameter is set to 0.7 and the digits eccentricity varies between 0 and 40 pixels. The presented examples correspond to strong eccentricity cases, when the target is hardly visible on the display (fig. \ref{fig:saccades}a), and almost invisible on the reconstructed input (fig. \ref{fig:saccades}b). The radial maps (fig. \ref{fig:saccades}c-d) respectively represent the actual and the predicted accuracy maps. The final focus is represented in fig. \ref{fig:saccades}e, with cases of classification success (fig. \ref{fig:saccades}A-B) and cases of classification failures (fig. \ref{fig:saccades}C-E).  
In the case of successful detection (fig. \ref{fig:saccades}A-B), the accuracy prediction is not perfect and the digit is not perfectly centered on the fovea. This ``close match'' however allows for a correct classification for the digit's pixels are fully present on the fovea. The case of fig. \ref{fig:saccades}B and \ref{fig:saccades}C
 is interesting for it shows two cases of a bimodal prediction, indicating that the network is capable of doing multiple detections at a single glance. The case of \ref{fig:saccades}C corresponds to a false detection, with the true target detected still, though with a lower intensity. The case of fig. \ref{fig:saccades}D is a ``close match'' detection that is not precise enough to correctly center the visual target. Not every pixel of the digit being visible on the fovea, the label prediction is mistaken.
The last failure case (fig. \ref{fig:saccades}E) corresponds to a correct detection that is harmed by a wrong label prediction, only due to the ``What'' classifier inherent error rate.  
% We observed that either the detection of the object's position was correct, thus allowing a classification proportional to the accuracy of the ``what'' pathway, either that the predicted accuracy map was wrong and generated a wrong classification with an accuracy at chance level. 


%: see Figure~\ref{fig:results}
\begin{figure}[t!]%%[p!]
	%\flushleft{\bf (A) \hspace{4.2cm} (B) \hspace{2cm} (C) \hspace{4cm} (D)\hspace{6cm}}
	\centering{\includegraphics[width=\linewidth]{fig-results-contrast.png}}
	% \includegraphics[]{../../2019-07-15_CNS/figures/CNS-saccade-8.png} % TRUE
	\caption{
		{\bf Effect of contrast and target eccentricity.} %
		The active vision agent is tested for different target eccentricities (in pixels) and different contrasts to estimate a final classification rate. Orange bars: accuracy of a central classifier ('No saccade') with respect to the target's eccentricity, averaged over 1,000 trials per eccentricity scale. Blue bars: Final classification rate after one saccade, as predicted by the ``Where'' pathway. % \if 0\ICANN{\color{blue} TODO: show three levels of noise Low (0.5) median (1.2) high (2). TODO: compare with Accuracy max (knowing the position)}\fi
		\label{fig:results}}%
\end{figure}%

By manipulating the SNR and the target eccentricity, one can precisely monitor the network detection and recognition capabilities, with a detection task ranging from `easy'' (small shift, strong contrast) to ``almost impossible'' (large shift, low contrast). The  digit recognition capability is systematically evaluated in Figure~\ref{fig:results} for different eccentricities and different contrasts. 
%in simulation as the average accuracy obtained at the landing of the predicted saccade (see). For each different visual display (a different digit at a different position with a different noise clutter), a retinocentric visual input is processed (figure \ref{fig:results}-A), providing a predicted accuracy map (figure \ref{fig:results}-B) that can be compared to the actual future accuracy. Then, a saccade is carried out based on the most probable position as computed from the predicted accuracy map (figure \ref{fig:results}-C), and the final accuracy is computed from the ``what'' pathway using LeNet model.
For 3 target contrasts conditions ranging from 0.3 to 0.7, and 10 different eccentricities ranging from 4 to 40 pixels, the final accuracy is tested on $1,000$ trials both on the initial central snippet and the final central snippet (read at the landing of the first saccade). 
The orange bars provide the initial classification rate (without saccade) and the blue bars provide the final classification rate (after one saccade) -- see figure \ref{fig:results}.  As expected, the accuracy decreases with the eccentricity, for the targets become less and less visible in the periphery. The decrease is rapid in the central classifier case: the accuracy drops to the baseline level
at approximately $20$ pixels away from the center of gaze. As expected, the saccade-driven accuracy has a much wider range, with a slow decrease up to the border of the visual display (40 pixels away from the center). The high contrast case (fig.~\ref{fig:results}A) shows the greatest difference, with an accuracy approaching 0.9 at the center and 0.6 at the periphery. This allows to recognize digits in one step in a majority of cases, up to the border of the image, from a very scarce peripheral information. This full covering of the 128$\times$128 image range is done at a much lesser cost than would be done by a systematic image scan, as in classic computer vision. With decreasing target contrast, a general decrease of the accuracy is observed, both at the center and at the periphery, with about 10\% decrease with a contrast of 0.5, and 40\% decrease with a contrast of 0.3. In addition, the proportion of false detections also increases with contrast decrease. At 40 pixels away from the center, the false detection rate is approximately 30\% for a contrast of 0.7, 50\% for a contrast of 0.5 and 70\% for a contrast of 0.3 (with a recognition close to the baseline in that case). The accuracy gain (difference between the initial and the final accuracy) is maximal for eccentricities ranging from 15 to 30 pixels. This optimal range reflects a peripheral region around the fovea where the target detection is possible, but not its identification. The visual agent knows \emph{where} the target is, without exactly knowing \emph{what} it is. 
More generally, this accuracy difference, that quantifies the benefit of active inference with respect to a central prior, can be interpreted as an approximation of the information gain provided by the ``Where'' pathway\footnote{with the true label log-posterior seen as a sample of the posterior entropy -- see eq.(\ref{eq:IG}).}.
% energy consumption

%The benefit of active inference can be enhanced by doing several saccades.   

As our saccade selection algorithm may implement the essential operations done in the ``Where'' pathway, the central classifier may also reflect the response of the ``What'' pathway, giving the potential category of the digit. It is therefore possible to compare the two accuracy estimates to chose the most appropriate action: it may be that the accuracy is best in the ``What'' pathway and in that case no saccade is produced. The decision frontier lies between the first and the second spatial scale, allowing to pursue micro-saccades in the close vicinity of the target (2-3 pixels), in order to achieve a perfect centering.  In the other decision case, the ''What'' accuracy can still be considered to update the ``Where'' accuracy. When extending our framework to several saccades, this would allow in particular to ``explain away" the current position of the fixation and the neighboring ones. Such heuristic gives a principled formulation of the inhibition of return mechanism which is an important aspect for modeling saccades~\citep{Itti01}. In particular, we predict that such a mechanism is dependent on the class of inputs, and would be different for searching for faces as compared to digits. 

\subsection{Quantitative role of parameters}
%: effect of contrast
To test the robustness of our framework, we repeated the same experiment at different signal-to-noise ratios (SNR) of the input images. Indeed, there is an interdependence of both pathways, and it is crucial to disentangle the relative efficiency of both sources of errors in the accuracy.
The result of Figure~\ref{fig:results}-D, correspond to a SNR of $0.7$ and we replicated the result for SNRs of $0.3$ and $0.5$. First, we re-iterated for each SNR the whole process, first by learning the ``what'' pathway, then the accuracy maps and finally the ``where'' pathway. We first observed that the accuracy map was scaled in the ``what'' pathway by the maximal accuracy (at the center of the image), respectively at a value of approximately $53\%$, $82\%$ and $92\%$ for SNRs of $0.3$, $0.5$ and $0.7$. Then, we observed that behavior of the ``where'' pathway was similar at the different SNRs values, yet with the scaling imposed by the ``what'' pathway. This shows the robustness of our framework to different levels of noise.

%: scanning of other parameters
In addition, we controlled that these results are robust to changes in an individual experimental or network parameters from the default parameters (see Figure~\ref{fig:params}). From the scan of each of these parameters, the following observations were remarkable. First we verified that accuracy decreased when\texttt{noise} increased and while the bandwidth of the noise imported weakly, the spatial frequency of the noise was an important factor. In particular, final accuracy was worst for $\texttt{sf\_0} \approx 0.07$, that is when the characteristic textures elements were close to the characteristic size of the objects. Second, we saw that the dimension of the ``where'' network was optimal for a dimensionality similar to that of the input but that this mattered weakly. The dimensionality of the log-polar map is more important. The analysis proved that an optimal accuracy was achieved when using a number of $24$ azimuthal directions. Indeed, a finer log-polar grid requires more epochs to converge and may result in an over-fitting phenomenon hindering the final accuracy. Such fine tuning of parameters may prove to be important in practical applications and to optimize the compromise between accuracy and compression. 
%=================================================================
%------------------------------%
%: see Figure~\ref{fig:params}
\begin{figure}[t!]%%[p!]
\centering{\includegraphics[width=\linewidth]{fig_params}}
\caption{
{\bf Quantitative role of parameters}: We show here variations of the average accuracy as a function of some free parameters of the model. All parameters of the presented model were tested, from the architecture of image generation, to the parameters of the neural network implementing the ``Where'' pathway (including meta-parameters of the learning paradigm). We show here the results which show the most significative impact on average accuracy. %
\A First, we tested some properties of the input, respectively from left to right: noise level (\texttt{noise}), mean spatial frequency of clutter \texttt{sf\_0} and bandwidth \texttt{B\_sf} of the clutter noise. This shows that average accuracy evolves with noise (see also Figure~\ref{fig:results} for an evolution as a function of eccentricity), but also to the characteristics of the noise clutter. In particular, there is a drop in accuracy whenever noise is of similar wavelength as digits, but which becomes less pronounced as the bandwidth increases. %
\B The accuracy also changes with the architecture of the foveated input as shown here by changing the number \texttt{N\_azimuth} of azimuth directions which are sampled in visual space. This shows a compromise between a rough azimuth representation and a large precision, which necessitates a longer training phase, such that the optimal number is around $20$ azimuth directions. %
\C Finally, we scanned parameters of the Deep Learning neural network. It shows that accuracy quickly converged after a characteristic time of approximately $25$ \texttt{epochs}. We then tested different values for the dimension of respectively the first (\texttt{dim1}) and second (\texttt{dim2}) hidden layers, showing weak changes in accuracy. %
\label{fig:params}}%
\end{figure}%
%%------------------------------%

% TODO : make a (minimal) psychophysics experiment= show an image as in figure 1, then in (ANS), make a 2AFC task by showing the true versus a random one -> web experiment using pavlovia?

%%
%%\newpage
%
%% !TEX root = paper.tex
% !TEX encoding = UTF-8 Unicode
% -*- coding: UTF-8; -*-
% vim: set fenc=utf-8
% !TEX spellcheck = en-US
\section{Discussion}
\label{sec:discussion}
%\subsection{Summary}
%## Main results:



In summary, we have proposed a visuo-motor action-selection model that implements a focal accuracy-seeking policy across the image. It relies on an interpretation of the Information Gain metric as a difference between central and peripheral accuracy processing. Each accuracy is predicted through separate processing pathways, namely the ``What'' pathway for the central pixels and the ``Where'' pathway for the periphery.  The comparison of both accuracies amounts either to select a saccade or to keep the eye focused at the center, so as to identify the label. The predicted accuracy map has, in our case, the role of a value-based action selection map, as it is the case in model-free reinforcement learning. However, it also owns a probabilistic interpretation that may be combined with concurrent accuracy predictions (such as the one done through the ``what'' pathway) to bring out more elaborate decision making which are relevant for visual search, such as inhibition of return for instance. 

Our main modelling assumption here is an \emph{accuracy-driven} monitoring of action, stating in short that the ventral classification accuracy drives the dorsal selection of action maps. {\color{magenta} Biological evidence for accuracy-map driven action selection? }

One crucial aspect of vision highlighted by our model is the importance of centering objects in recognition. Despite the robust translation invariance observed on the ``What'' pathway, there is small radius of 2-3 pixels around the target's center that needs to be respected to maximize the classification accuracy. This relates to the idea of finding an absolute referential for an object, for which the recognition is easier. If the center of fixation is fixed, the log-polar encoding of an object class shows invariance to both rotation and scale~\citep{Traver10}. Incorporating this scale and rotation invariance may thus extend the recognition capabilities of the model.


Compared to classical architectures, this approach is energy-efficient. It encompasses a full log-polar processing pathway which induces a high compression rate similar to that performed by retina and V1 encoding up to the action selection level. This finally provides an effective sub-linear visual search scheme (compared to the linear time necessary to scan all positions on a regular grid), that may allow to detect an object in large visual environments at little cost. This should be particularly beneficial when the computing resources are under constraint, such as for drones or mobile robots. 

One limit of our model is the simplicity of the generative model for inputs. This simplicity was important to fully assess the possibility and robustness of implementing a ``where'' module which predicts the map of accuracies from the (degraded) log-polar transformed retinal image.
In future perspectives, more elaborate image categorization, such as the ones performed on natural images using deep convolutional nets, could be considered by replacing the current ``what'' module with a more elaborate one. 
%By preserving a probabilistic interpretation in bio-realistic action selection, 


Finally, our model relies on a strong idealization, assuming the presence of an unique target. The presence of many targets in a scene should be addressed, which amounts to sequentially select targets, in combination with implementing an inhibition of return mechanism. 
%Moreover, inhibition of return mechanism could envisioned by preserving a probabilistic interpretation in bio-realistic action selection. 
This would generate more realistic visual scan-paths over images. %This could be used to provide realistic priors over action selection maps.  
%In particular, identified regions of interest may then be compared with the baseline bottom-up approaches, such as the low-level feature-based saliency maps~\citep{Itti01}. 
Actual visual scan path over images could also be used to provide priors over action selection maps that should improve realism.  %
%It may indeed be possible to consider , and 
Identified regions of interest may then be compared with the baseline bottom-up approaches, such as the low-level feature-based saliency maps~\citep{Itti01}. 
Maximizing the Information Gain over multiple targets needs to be envisioned with a more refined probabilistic framework, including mutual exclusion over overt and covert targets. How the brain may combine and integrate these various probabilities is still an open question, that amounts to the fundamental binding problem. %: How is it possible to meaningfully combine independently extracted features.


%%
%%\newpage
%% \section*{General case: Visual information gain maximization}\label{sec:case1}

Consider a view $\boldsymbol{x}$ generated from a target $\boldsymbol{y}$ viewed at retinocentric position $\boldsymbol{u}$. 

Consider first that :
\begin{itemize}
	\item The generative model  $p(X|\boldsymbol{y}, \boldsymbol{u})$ is known
	\item The retinocentric position  $\boldsymbol{u}$ is known.
	\item The view $\boldsymbol{x}$ is known.
	\item The target category  $\boldsymbol{y}$ is unknown.
\end{itemize} 



The question comes how to choose the new retinocentric position $\boldsymbol{u}'$ in order to maximize the \emph{mutual information} between $\boldsymbol{x}|\boldsymbol{u}$ (current view) and $\boldsymbol{x}'|\boldsymbol{u}'$ (future view).

In general, the visual Information Gain between two visual fields $\boldsymbol{x}|\boldsymbol{u}$  and $\boldsymbol{x}'|\boldsymbol{u}'$ is:

\begin{align*}
\text{IG}(\boldsymbol{x}|\boldsymbol{u}; \boldsymbol{x}'| \boldsymbol{u}') 
&= -\log p(\boldsymbol{x}|\boldsymbol{u}) 
+ \log p(\boldsymbol{x}|\boldsymbol{u}, \boldsymbol{x}', \boldsymbol{u}')
\end{align*}

\paragraph{Information Gain Lower Bound}
Consider now that given  $\boldsymbol{x}$ and $\boldsymbol{u}$, the target category  $\boldsymbol{y}$ can be \emph{inferred} using Bayes rule, i.e.:
$$ P(Y|\boldsymbol{x}, \boldsymbol{u}) \propto  P(\boldsymbol{x}|Y, \boldsymbol{u}) $$
Then, it can be shown (see \cite{dauce2018}) that :
$$\text{IG}(\boldsymbol{x}|\boldsymbol{u}; \boldsymbol{x}'| \boldsymbol{u}') \geq \mathbb{E}_{\boldsymbol{y}\sim p(Y|\boldsymbol{x}, \boldsymbol{u})} \left[\log p(\boldsymbol{y}|\boldsymbol{x}', \boldsymbol{u}') - \log(\pi(\boldsymbol{y})) \right]$$
with  $\pi(\boldsymbol{y})$ the prior over the $\boldsymbol{y}$'s .
When the prior is uniform, the information gain lower bound (IGLB) simplifies to $\mathbb{E}_{\boldsymbol{y}\sim p(Y|\boldsymbol{x}, \boldsymbol{u})} \left[\log p(\boldsymbol{y}|\boldsymbol{x}', \boldsymbol{u}')\right] + c$, with $c$ a constant.

\paragraph{Predictive approach}
One can adopt a \emph{predictive} approach to choose the new eye orientation $\boldsymbol{e}'$:
\begin{itemize}
	\item First choose a new retinocentric position $\boldsymbol{u}'$ that will maximize the  information gain.
	\item Then choose $\boldsymbol{e}'$ such that $$\boldsymbol{z} - \boldsymbol{e}' = \boldsymbol{u}'$$ i.e. $$\boldsymbol{e}' = \boldsymbol{e} + \boldsymbol{u} - \boldsymbol{u}'$$
\end{itemize}

The predictive approach needs three predictive steps:
\begin{itemize}
	\item $p(Y|\boldsymbol{x}, \boldsymbol{u})$ is the current posterior over the target category inferred from the current observation,
	\item $\boldsymbol{x}'\sim p(X|\boldsymbol{y},\boldsymbol{u}')$ is the predicted view generated by the model assuming that the target $\boldsymbol{y}$ is seen from from $\boldsymbol{u}'$,
	\item and $p(\boldsymbol{y}|\boldsymbol{x}', \boldsymbol{u}')$ is the predicted posterior for   assumption $\boldsymbol{y}$, given $\boldsymbol{x}'$ and $\boldsymbol{u}'$.
\end{itemize}

Then the optimal new retinocentric position is:
\begin{align*}
\hat{\boldsymbol{u}}' &= \underset{\boldsymbol{u}' }{\text{ argmax }} 
 \mathbb{E}_{\boldsymbol{y}\sim p(Y|\boldsymbol{x}, \boldsymbol{u})}  
 \left[\mathbb{E}_{ \boldsymbol{x}' \sim p(X|\boldsymbol{y}, \boldsymbol{u}')}
 \left[\log p(\boldsymbol{y}|\boldsymbol{x}', \boldsymbol{u}')\right]\right]\\
  %&= \underset{\boldsymbol{u}' \in \mathcal{U}}{\text{ argmax }} A(\boldsymbol{u}'|\boldsymbol{x}, \boldsymbol{u})
\end{align*}

Taking $\delta \boldsymbol{e} = \boldsymbol{u} - \boldsymbol{u}'$, the optimal eye displacement is: 
\begin{align*}
\widehat{\delta\boldsymbol{e}} &= \underset{\delta\boldsymbol{e} }{\text{ argmax }} 
\mathbb{E}_{\boldsymbol{y}\sim p(Y|\boldsymbol{x}, \boldsymbol{u})}  
\left[\mathbb{E}_{ \boldsymbol{x}' \sim p(X|\boldsymbol{y}, \boldsymbol{u}- \delta \boldsymbol{e})}
\left[\log p(\boldsymbol{y}|\boldsymbol{x}', \boldsymbol{u}-\delta\boldsymbol{e})\right]\right]\\
%&= \underset{\delta\boldsymbol{e}}{\text{ argmax }} A(\delta\boldsymbol{e}|\boldsymbol{x}, \boldsymbol{u})
\end{align*}

%For each possible target identity $\boldsymbol{y}$, $A_{\boldsymbol{y}}(\boldsymbol{u}') = \mathbb{E}_{ \boldsymbol{x}' \sim p(X|\boldsymbol{y}, \boldsymbol{u}')}
%\left[\log p(\boldsymbol{y}|\boldsymbol{x}', \boldsymbol{u}')\right]$ is the \emph{class-specific log posterior} map and 




%Then:
%$$\tilde{A}_{\boldsymbol{y}}(\boldsymbol{u}') = \mathbb{E}_{ \boldsymbol{x}' \sim p(X|\boldsymbol{y}, \boldsymbol{u}')}$$

\emph{(TODO : Attention il faudrait à partir de maintenant une carte qui moyenne les log posteriors car l'espérance du log n'est pas  égale au log de l'espérance, i.e. $r_{\theta}^{\text{log}}(\boldsymbol{u}|q) = \mathbb{E}_{\boldsymbol{y}\sim q(Y)} \left[\mathbb{E}_{ \boldsymbol{x} \sim p(X|\boldsymbol{y}, \boldsymbol{u})}  \log p_\theta(\boldsymbol{y}|\boldsymbol{x}, \boldsymbol{u}) \right]$).}
\newline

%
%
%%%%-----------------------------------------------------------------
%%{\bf References} \\
%
%\printbibliography[heading=subbibliography]
%
%
%
%\end{document}


% Title must be 250 characters or less.
\begin{flushleft}
{\Large
\textbf\newline{
\Title
} % Please use "sentence case" for title and headings (capitalize only the first word in a title (or heading), the first word in a subtitle (or subheading), and any proper nouns).
}
\newline 
% Insert author names, affiliations and corresponding author email (do not include titles, positions, or degrees).
\\
\AuthorED, %\textsuperscript{1},
\AuthorPA, %\textsuperscript{2},
\AuthorLP%\textsuperscript{2,*}
\\
\bigskip
\Address
% \textbf{1} \AddressED
% \\
% \textbf{2} \AddressLP
% \\
\bigskip

% Insert additional author notes using the symbols described below. Insert symbol callouts after author names as necessary.
%
% Remove or comment out the author notes below if they aren't used.
%
% Primary Equal Contribution Note
%\Yinyang These authors contributed equally to this work.

% Additional Equal Contribution Note
% Also use this double-dagger symbol for special authorship notes, such as senior authorship.
%\ddag These authors also contributed equally to this work.

% Current address notes
%\textcurrency Current Address: Dept/Program/Center, Institution Name, City, State, Country % change symbol to "\textcurrency a" if more than one current address note
% \textcurrency b Insert second current address
% \textcurrency c Insert third current address

% Deceased author note
%\dag Deceased

% Group/Consortium Author Note
%\textpilcrow Membership list can be found in the Acknowledgments section.

% Use the asterisk to denote corresponding authorship and provide email address in note below.
* \EmailLP

\end{flushleft}
% Please keep the abstract below 300 words
\section*{Abstract}
\Abstract

% Please keep the Author Summary between 150 and 200 words
% Use first person. PLOS ONE authors please skip this step.
% Author Summary not valid for PLOS ONE submissions.
\section*{Author summary}
\AuthorSummary
\linenumbers

% Use "Eq" instead of "Equation" for equation citations.

% !TEX root = paper.tex
% !TEX encoding = UTF-8 Unicode
% -*- coding: UTF-8; -*-
% vim: set fenc=utf-8
% !TEX spellcheck = en-US
\section{Introduction}
\label{sec:intro}
\paragraph{Problem statement.}
%------------------------------%
%: see Figure~\ref{fig:intro}
\begin{figure}[t!]%[b!]%%[p!]
	\centering{	\includegraphics[width=\linewidth]{fig_intro}} %
	\caption{%
		{\bf Problem setting}: In generic, ecological settings, the visual system faces a tricky problem when searching for one target (from a class of targets) in a cluttered environment. It is synthesized in the following experiment: %
		\A After a fixation period \FIX\ of $200~\ms$, an observer is presented with a luminous display \DIS\ showing a single target from a known class (here digits) and at a random position. The display is presented for a short period of $500~\ms$ (light shaded area in B), that is enough to perform at most one saccade on the potential target (\SAC , here successful). Finally, the observer has to identify the digit by a keypress \ANS . %
		\B Prototypical trace of a saccadic eye movement to the target position. In particular, we show the fixation window \FIX\ and the temporal window during which a saccade is possible (green shaded area). %
		\C Simulated reconstruction of the visual information from the (interoceptive) retinotopic map at the onset of the display \DIS\ and after a saccade \SAC , the dashed red box indicating the visual area of the ``what'' pathway. In contrast to an exteroceptive representation (see A), this demonstrates that the position of the target has to be inferred from a degraded (sampled) image. In particular, the configuration of the display is such that by adding clutter and reducing the size of the digit, it may become necessary to perform a saccade to be able to identify the digit. The computational pathway mediating the action has to infer the location of the target \emph{before seeing it}, that is, before being able to actually identify the target's category from a central fixation. %
		\label{fig:intro}}%
\end{figure}%
%%------------------------------%

The promise of artificial vision to identify objects in natural images is ever increasing. Image processing algorithms recently outreached the performance of human observers in specific image categorization tasks~\citep{He15}. Initially trained on energy greedy, high performance computers, they are now designed to work on more common hardware such as desktop computers with dedicated GPU hardware~\citep{Sandler18}. However, these algorithms are still far from human performances, even for simple tasks. Take for instance the case of an encounter with a friend in a crowded café. To catch the moment at which she arrives, you need to visually search for her face despite the sensory clutter in the visual field. To do so, you need to scan relevant parts of the visual scene with your gaze. Doing a saccade at these locations will allow you to recognize your friend. The main difficulty of this task is to learn to categorize this particular object class given all possible spatial configurations and respective geometrical visual transformations. 

This visual search experience can be formalized and simplified in a way reminiscent to classical psychophysical experiments: an observer is asked to classify digits (for instance as taken from the MNIST database) as they are shown on a computer display. However, these digits can be placed at random positions on the display, and visual clutter is added as a background to the image (see Figure~\ref{fig:intro}-A). This opens the possibility that the position of the object may be detected in the clutter without being identified in the first place (see Figure~\ref{fig:intro}-C). This defines more precisely our problem: how do we localize an object in a large image while knowing \emph{a priori} its category but not its identity? This generic visual search problem is of broad interest in machine learning, computer vision and robotics, but also in neuroscience, as it speaks to the mechanisms underlying foveation and more generally to low-level attention mechanisms.

Inherent to this problem is the combinatorial explosion implied by an increasing number of parameters. State-of-the art classification architectures consequently contain many millions parameters with subsequent energy consumption increase while still handling relatively small images. This introduces a trade-off between efficiency and average accuracy, for instance in autonomous driving such that the algorithm is fast enough to detect visual objects in a glance while running on resource-constrained devices like embedded devices. Globally, this performance is still lower than that of humans. Indeed, the human visual system can perform such a feat both rapidly, --~in less than 100 ms~\citep{Kirchner06}~-- and at a low energy cost ($<5~W$). On top of that, it is mostly self-organized, robust to visual transforms or lighting conditions and can learn with a few examples. If many different anatomical features may explain this efficiency, a main difference lies in the fact that its sensor (the retina) combines a non homogeneous sampling of the world with the capacity to rapidly change its center of fixation. Indeed, on the one hand, the retina is composed of two separate systems: a central, high definition fovea (a disk of about 6 degrees of diameter in visual angle around the center of gaze) and a large, lower definition peripheral area. On the other hand, the retina is attached on the back of the eye which is capable of low latency, high speed eye movements. In particular, saccades allow for efficient changes of the position of the center of gaze: they take about $200~\ms$ to initiate, last about $200~\ms$ and usually reach a maximum velocity of approx 600 degrees per second. This behavior is prevalent during our lifetime (about a saccade every 2-3 seconds, that is, almost a billion saccade in a lifetime). The interplay of those two features allows human observers to engage in an integrated action perception loop which sequentially scans and analyses the different parts of the image.
%It is one type of active inference~\citep{Friston12} (see below) and we will envision herein how to incorporate it to classical computer vision schemes.
% (1 / 2.5 * 3600 * 24 * 365 * 75 = 946080000.0 ~= .95e9) X (wakeful + REM = .66)
%
\paragraph{State of the art.}

To take advantage of this visuomotor behavior, it is of particular importance to understand both its computational and neurophysiological principles. First, the joint problem of target localization and identification is a classical problem of visual search in computer vision. It is very general and may address apparently simple questions such as ``find the green bottle on the table''. 
When restricted to a mere ``feature search''~\citep{Treisman80}, many solutions are proposed. Notably, recent advances in deep-learning have provided efficient models such as faster-RCNN~\citep{Ren17} or YOLO~\citep{Redmon15}. 
This last implementation is particularly interesting for our sake as it predicts in the image the probability of proposed bounding boxes around visual objects. While rapid, the number of boxes greatly increases with image size and necessitates dedicated hardware. 
In parallel, when limited to a few objects of interest in the image, this strategy amounts to a classical problem in neuroscience, that is, the transformation of a luminous image into a saliency map~\citep{Itti01}, essential to understand and predict saccades, but also to serve as phenomenological models of attention. The saliency approach was recently extended using deep learning to estimate saliency maps over large databases of natural images~\citep{Kummerer16}. While these methods are efficient at predicting the probability of fixation, they miss an essential component in the action perception loop: they operate on the full image while the retina operates on the non-uniform, foveated sampling of visual space (see Figure~\ref{fig:intro}-B). 
Herein, we believe that this fact is an essential factor to reproduce and understand this active vision process.

In contrast to phenomenological (or ``bottom-up'') approaches, models of active vision~\citep{Najemnik05,Butko2010infomax,Friston12} provide the ground principles of saccadic exploration. In general, they assume the existence of a generative model from which both the target position and category can be inferred through active sampling. This comes from the constraint that the visual sensor is foveated but can generate a saccade. 
Several studies are relevant to our endeavor. First, one can consider optimal strategies to solve the problem of the visual search of a target~\citep{Najemnik05}. In a setting similar to that presented in Figure~\ref{fig:intro}-A, where the target is an oriented edge and the background is defined as pink noise, authors show first that a Bayesian ideal observer comes out with an optimal strategy, and second that human observers are close to that optimal performance. Though well predicting sequences of saccades in a perception action loop, this model is limited by the simplicity of the display (elementary edges added on stationary noise, a finite number of locations on a discrete grid) and by the abstract level of modeling. Despite these (inevitable) simplifications, this study could successfully predict some key characteristics of visual scanning such as the trade-off between memory content and rapidity. Looking more closely at neurophysiology, the study of~\citep{Samonds18} allows to go further in understanding the interplay between saccadic behavior and the statistics of the input. In this study, authors were able to manipulate the size of the saccades by monitoring key properties of the presented (natural) images. For instance, smaller images generate smaller saccades. Interestingly, they also predicted the size of saccades for different species, including mice which lack a foveal region, from the size of visual receptive fields. One key prediction of this study which is relevant for our problem is the fact that saccades seem optimal to \emph{a priori} decorrelate the visual input, that is, to minimize redundancy in the sequence of generated saccades, knowing the statistics of the visual inputs.

A further modeling perspective is provided by~\citep{Friston12}. In this setup, a full description of the visual world is used as a generative process. An agent is completely described by giving the generative model governing the dynamics of its internal beliefs and is interacting with this image by scanning it through a foveated sensor, just as described in Figure~\ref{fig:intro}. Thus, equipping the agent with the ability to actively sample the visual world %enables to explore the idea that actions (saccadic eye movements) are 
allows to interpret saccades as optimal experiments, by which the agent seeks to confirm predictive models of the (hidden) world. One key ingredient to this process is the (internal) representation of counterfactual predictions, that is, the probable consequences of possible hypothesis as they would be realized into actions (here, saccades). Following such an active inference scheme~\citep{Mirza18} numerical simulations reproduce sequential eye movements that fit well with empirical data. %Compared to~\citet{Najemnik05}, 
Saccades %are not the output of a value-based cost function, but 
are here a consequence of an active seek for the agent to minimize the uncertainty about his beliefs, knowing his priors on the generative model of the visual world. 

\paragraph{Outline.}
Stemming from the active vision general principles, our aim is to produce a principled model that may both explain the essential features of human vision and provide ways toward efficient computer implementations. We also aim at reunifying the fragmentation of the many different approaches respective to their fields (Machine learning, neuroscience, robotics), and envisage an integrated computational model of foveated active vision. It is known that inverting a generative model over a large (one-step ahead) hypothesis space of all possible saccades is computationally-intensive. % (think for instance of face category as a very large categorical space over a large visual transformation space) with no obvious neurophysiological counterpart. (see Figure~\ref{fig:intro}-C)
%Although we similarly include a generative process of the visual world,
Herein, we hypothesize that complex combinatorial inferences can be replaced by separate pathways, i.e. the spatial (``where'') and categorical (``what'') pathways, whose knowledge is combined to infer optimal eye displacements and subsequent identification of the target. 
%as conatining {\bf (containing??)} images of a handwritten random digit (drawn from the MNIST database) at a random position and embedded in a cluttered noise . 
We will thus define an agent equipped with a foveated sensor and with the ability to actively scan the visual image, %. % as defined by a generative (internal) modelWe will use this constraint as an asset 
%to which also contributes to minimizing the overall computational cost of finding a target. 
%Taking such priors, we 
learn an optimal behavior startegy and explore its key properties.

This paper is organized as follows: After this introduction, we define the principles underlying accuracy-based saccadic control in section~\ref{sec:principles}. We first define notations, variables and equations for the generative process governing the experiment and the generative model for the active vision agent. In particular, we derive our method to simplify the learning of an optimal agent given these definitions. In section \ref{sec:implementation}, implementation details are given, providing ways to reproduce our results. In section~\ref{sec:results}, preliminary results of numerical simulations of the agent are presented, demonstrating the applicability of this framework to different task complexity levels. This allows us to derive some limits of the agent and, as in~\citep{Najemnik05}, we draw some analogies with biologically observed eye movements. Finally, in section~\ref{sec:discussion}, we summarize these results in comparison with other similar schemes. We conclude by showing the relative advantages of using this active inference approach.


% !TEX root = paper.tex
% !TEX encoding = UTF-8 Unicode
% -*- coding: UTF-8; -*-
% vim: set fenc=utf-8
% !TEX spellcheck = en-US
%=================================================================
\section{Results}
\label{sec:results}
%=================================================================
%------------------------------%

\begin{figure}[t!]%%[p!]
	%\flushleft{\bf (A) \hspace{4.2cm} (B) \hspace{2cm} (C) \hspace{4cm} (D)\hspace{6cm}}
	\centering{{\bf a.} \hspace{1.6cm} {\bf b.} \hspace{1.6cm} {\bf c.} \hspace{1.6cm} {\bf d.} \hspace{1.6cm} {\bf e.}}
	\centering{\A\includegraphics[width=.9\linewidth]{CNS-saccade-8.png}}
	\centering{\B\includegraphics[width=.9\linewidth]{CNS-saccade-20.png}}
	\centering{\C\includegraphics[width=.9\linewidth]{CNS-saccade-46.png}}
	\centering{\D\includegraphics[width=.9\linewidth]{CNS-saccade-47.png}}
	\centering{\E\includegraphics[width=.9\linewidth]{CNS-saccade-32.png}}		% \includegraphics[]{../../2019-07-15_CNS/figures/CNS-saccade-8.png} % TRUE
	\caption{
		{\A-- \E Active vision samples after training}. \A-- \B  classification success samples. \C-- \E classification failure samples. Digit contrast set to 0.7.  From left to right :
		{\bf a.} The initial 128$\times$128 visual display, with blue cross giving the center of gaze.  
		The visual input is retinotopically transformed and sent to the multi-layer neural network implementing the ``Where'' pathway. 
		{\bf b.} Magnified reconstruction of the  visual input, as it shows off from the primary visual features through an inverse log-polar transform. 
		{\bf c.-d.} Color-coded radial representation of the output accuracy maps, with dark violet for the lower accuracies, and yellow for the higher accuracies. The network output ('Predicted') is visually compared with the ground truth ('True'). %
		{\bf e.} $28 \times 28$ central snippet as extracted from the visual display after doing a saccade, with label prediction and success flag in the title. 
		\label{fig:saccades}}%
\end{figure}%

%%------------------------------%
%=================================================================
\subsection{Inferring the target label from active visual search}
%=================================================================
After training, the ``Where'' pathway is now capable to predict an accuracy map, whose maximal argument drives the eye toward a new viewpoint. There, a central snippet is extracted, that is processed through the ``What'' pathway, allowing to predict the digit's label. Examples of this simple open loop sequence are presented in figure \ref{fig:saccades}, when the digits contrast parameter is set to 0.7 and the digits eccentricity varies between 0 and 40 pixels. The presented examples correspond to strong eccentricity cases, when the target is hardly visible on the display (fig. \ref{fig:saccades}a), and almost invisible on the reconstructed input (fig. \ref{fig:saccades}b). The radial maps (fig. \ref{fig:saccades}c-d) respectively represent the actual and the predicted accuracy maps. The final focus is represented in fig. \ref{fig:saccades}e, with cases of classification success (fig. \ref{fig:saccades}A-B) and cases of classification failures (fig. \ref{fig:saccades}C-E).  
In the case of successful detection (fig. \ref{fig:saccades}A-B), the accuracy prediction is not perfect and the digit is not perfectly centered on the fovea. This ``close match'' however allows for a correct classification for the digit's pixels are fully present on the fovea. The case of fig. \ref{fig:saccades}B and \ref{fig:saccades}C
 is interesting for it shows two cases of a bimodal prediction, indicating that the network is capable of doing multiple detections at a single glance. The case of \ref{fig:saccades}C corresponds to a false detection, with the true target detected still, though with a lower intensity. The case of fig. \ref{fig:saccades}D is a ``close match'' detection that is not precise enough to correctly center the visual target. Not every pixel of the digit being visible on the fovea, the label prediction is mistaken.
The last failure case (fig. \ref{fig:saccades}E) corresponds to a correct detection that is harmed by a wrong label prediction, only due to the ``What'' classifier inherent error rate.  
% We observed that either the detection of the object's position was correct, thus allowing a classification proportional to the accuracy of the ``what'' pathway, either that the predicted accuracy map was wrong and generated a wrong classification with an accuracy at chance level. 


%: see Figure~\ref{fig:results}
\begin{figure}[t!]%%[p!]
	%\flushleft{\bf (A) \hspace{4.2cm} (B) \hspace{2cm} (C) \hspace{4cm} (D)\hspace{6cm}}
	\centering{\includegraphics[width=\linewidth]{fig-results-contrast.png}}
	% \includegraphics[]{../../2019-07-15_CNS/figures/CNS-saccade-8.png} % TRUE
	\caption{
		{\bf Effect of contrast and target eccentricity.} %
		The active vision agent is tested for different target eccentricities (in pixels) and different contrasts to estimate a final classification rate. Orange bars: accuracy of a central classifier ('No saccade') with respect to the target's eccentricity, averaged over 1,000 trials per eccentricity scale. Blue bars: Final classification rate after one saccade, as predicted by the ``Where'' pathway. % \if 0\ICANN{\color{blue} TODO: show three levels of noise Low (0.5) median (1.2) high (2). TODO: compare with Accuracy max (knowing the position)}\fi
		\label{fig:results}}%
\end{figure}%

By manipulating the SNR and the target eccentricity, one can precisely monitor the network detection and recognition capabilities, with a detection task ranging from `easy'' (small shift, strong contrast) to ``almost impossible'' (large shift, low contrast). The  digit recognition capability is systematically evaluated in Figure~\ref{fig:results} for different eccentricities and different contrasts. 
%in simulation as the average accuracy obtained at the landing of the predicted saccade (see). For each different visual display (a different digit at a different position with a different noise clutter), a retinocentric visual input is processed (figure \ref{fig:results}-A), providing a predicted accuracy map (figure \ref{fig:results}-B) that can be compared to the actual future accuracy. Then, a saccade is carried out based on the most probable position as computed from the predicted accuracy map (figure \ref{fig:results}-C), and the final accuracy is computed from the ``what'' pathway using LeNet model.
For 3 target contrasts conditions ranging from 0.3 to 0.7, and 10 different eccentricities ranging from 4 to 40 pixels, the final accuracy is tested on $1,000$ trials both on the initial central snippet and the final central snippet (read at the landing of the first saccade). 
The orange bars provide the initial classification rate (without saccade) and the blue bars provide the final classification rate (after one saccade) -- see figure \ref{fig:results}.  As expected, the accuracy decreases with the eccentricity, for the targets become less and less visible in the periphery. The decrease is rapid in the central classifier case: the accuracy drops to the baseline level
at approximately $20$ pixels away from the center of gaze. As expected, the saccade-driven accuracy has a much wider range, with a slow decrease up to the border of the visual display (40 pixels away from the center). The high contrast case (fig.~\ref{fig:results}A) shows the greatest difference, with an accuracy approaching 0.9 at the center and 0.6 at the periphery. This allows to recognize digits in one step in a majority of cases, up to the border of the image, from a very scarce peripheral information. This full covering of the 128$\times$128 image range is done at a much lesser cost than would be done by a systematic image scan, as in classic computer vision. With decreasing target contrast, a general decrease of the accuracy is observed, both at the center and at the periphery, with about 10\% decrease with a contrast of 0.5, and 40\% decrease with a contrast of 0.3. In addition, the proportion of false detections also increases with contrast decrease. At 40 pixels away from the center, the false detection rate is approximately 30\% for a contrast of 0.7, 50\% for a contrast of 0.5 and 70\% for a contrast of 0.3 (with a recognition close to the baseline in that case). The accuracy gain (difference between the initial and the final accuracy) is maximal for eccentricities ranging from 15 to 30 pixels. This optimal range reflects a peripheral region around the fovea where the target detection is possible, but not its identification. The visual agent knows \emph{where} the target is, without exactly knowing \emph{what} it is. 
More generally, this accuracy difference, that quantifies the benefit of active inference with respect to a central prior, can be interpreted as an approximation of the information gain provided by the ``Where'' pathway\footnote{with the true label log-posterior seen as a sample of the posterior entropy -- see eq.(\ref{eq:IG}).}.
% energy consumption

%The benefit of active inference can be enhanced by doing several saccades.   

As our saccade selection algorithm may implement the essential operations done in the ``Where'' pathway, the central classifier may also reflect the response of the ``What'' pathway, giving the potential category of the digit. It is therefore possible to compare the two accuracy estimates to chose the most appropriate action: it may be that the accuracy is best in the ``What'' pathway and in that case no saccade is produced. The decision frontier lies between the first and the second spatial scale, allowing to pursue micro-saccades in the close vicinity of the target (2-3 pixels), in order to achieve a perfect centering.  In the other decision case, the ''What'' accuracy can still be considered to update the ``Where'' accuracy. When extending our framework to several saccades, this would allow in particular to ``explain away" the current position of the fixation and the neighboring ones. Such heuristic gives a principled formulation of the inhibition of return mechanism which is an important aspect for modeling saccades~\citep{Itti01}. In particular, we predict that such a mechanism is dependent on the class of inputs, and would be different for searching for faces as compared to digits. 

\subsection{Quantitative role of parameters}
%: effect of contrast
To test the robustness of our framework, we repeated the same experiment at different signal-to-noise ratios (SNR) of the input images. Indeed, there is an interdependence of both pathways, and it is crucial to disentangle the relative efficiency of both sources of errors in the accuracy.
The result of Figure~\ref{fig:results}-D, correspond to a SNR of $0.7$ and we replicated the result for SNRs of $0.3$ and $0.5$. First, we re-iterated for each SNR the whole process, first by learning the ``what'' pathway, then the accuracy maps and finally the ``where'' pathway. We first observed that the accuracy map was scaled in the ``what'' pathway by the maximal accuracy (at the center of the image), respectively at a value of approximately $53\%$, $82\%$ and $92\%$ for SNRs of $0.3$, $0.5$ and $0.7$. Then, we observed that behavior of the ``where'' pathway was similar at the different SNRs values, yet with the scaling imposed by the ``what'' pathway. This shows the robustness of our framework to different levels of noise.

%: scanning of other parameters
In addition, we controlled that these results are robust to changes in an individual experimental or network parameters from the default parameters (see Figure~\ref{fig:params}). From the scan of each of these parameters, the following observations were remarkable. First we verified that accuracy decreased when\texttt{noise} increased and while the bandwidth of the noise imported weakly, the spatial frequency of the noise was an important factor. In particular, final accuracy was worst for $\texttt{sf\_0} \approx 0.07$, that is when the characteristic textures elements were close to the characteristic size of the objects. Second, we saw that the dimension of the ``where'' network was optimal for a dimensionality similar to that of the input but that this mattered weakly. The dimensionality of the log-polar map is more important. The analysis proved that an optimal accuracy was achieved when using a number of $24$ azimuthal directions. Indeed, a finer log-polar grid requires more epochs to converge and may result in an over-fitting phenomenon hindering the final accuracy. Such fine tuning of parameters may prove to be important in practical applications and to optimize the compromise between accuracy and compression. 
%=================================================================
%------------------------------%
%: see Figure~\ref{fig:params}
\begin{figure}[t!]%%[p!]
\centering{\includegraphics[width=\linewidth]{fig_params}}
\caption{
{\bf Quantitative role of parameters}: We show here variations of the average accuracy as a function of some free parameters of the model. All parameters of the presented model were tested, from the architecture of image generation, to the parameters of the neural network implementing the ``Where'' pathway (including meta-parameters of the learning paradigm). We show here the results which show the most significative impact on average accuracy. %
\A First, we tested some properties of the input, respectively from left to right: noise level (\texttt{noise}), mean spatial frequency of clutter \texttt{sf\_0} and bandwidth \texttt{B\_sf} of the clutter noise. This shows that average accuracy evolves with noise (see also Figure~\ref{fig:results} for an evolution as a function of eccentricity), but also to the characteristics of the noise clutter. In particular, there is a drop in accuracy whenever noise is of similar wavelength as digits, but which becomes less pronounced as the bandwidth increases. %
\B The accuracy also changes with the architecture of the foveated input as shown here by changing the number \texttt{N\_azimuth} of azimuth directions which are sampled in visual space. This shows a compromise between a rough azimuth representation and a large precision, which necessitates a longer training phase, such that the optimal number is around $20$ azimuth directions. %
\C Finally, we scanned parameters of the Deep Learning neural network. It shows that accuracy quickly converged after a characteristic time of approximately $25$ \texttt{epochs}. We then tested different values for the dimension of respectively the first (\texttt{dim1}) and second (\texttt{dim2}) hidden layers, showing weak changes in accuracy. %
\label{fig:params}}%
\end{figure}%
%%------------------------------%

% TODO : make a (minimal) psychophysics experiment= show an image as in figure 1, then in (ANS), make a 2AFC task by showing the true versus a random one -> web experiment using pavlovia?


% !TEX root = paper.tex
% !TEX encoding = UTF-8 Unicode
% -*- coding: UTF-8; -*-
% vim: set fenc=utf-8
% !TEX spellcheck = en-US
\section{Discussion}
\label{sec:discussion}
%\subsection{Summary}
%## Main results:



In summary, we have proposed a visuo-motor action-selection model that implements a focal accuracy-seeking policy across the image. It relies on an interpretation of the Information Gain metric as a difference between central and peripheral accuracy processing. Each accuracy is predicted through separate processing pathways, namely the ``What'' pathway for the central pixels and the ``Where'' pathway for the periphery.  The comparison of both accuracies amounts either to select a saccade or to keep the eye focused at the center, so as to identify the label. The predicted accuracy map has, in our case, the role of a value-based action selection map, as it is the case in model-free reinforcement learning. However, it also owns a probabilistic interpretation that may be combined with concurrent accuracy predictions (such as the one done through the ``what'' pathway) to bring out more elaborate decision making which are relevant for visual search, such as inhibition of return for instance. 

Our main modelling assumption here is an \emph{accuracy-driven} monitoring of action, stating in short that the ventral classification accuracy drives the dorsal selection of action maps. {\color{magenta} Biological evidence for accuracy-map driven action selection? }

One crucial aspect of vision highlighted by our model is the importance of centering objects in recognition. Despite the robust translation invariance observed on the ``What'' pathway, there is small radius of 2-3 pixels around the target's center that needs to be respected to maximize the classification accuracy. This relates to the idea of finding an absolute referential for an object, for which the recognition is easier. If the center of fixation is fixed, the log-polar encoding of an object class shows invariance to both rotation and scale~\citep{Traver10}. Incorporating this scale and rotation invariance may thus extend the recognition capabilities of the model.


Compared to classical architectures, this approach is energy-efficient. It encompasses a full log-polar processing pathway which induces a high compression rate similar to that performed by retina and V1 encoding up to the action selection level. This finally provides an effective sub-linear visual search scheme (compared to the linear time necessary to scan all positions on a regular grid), that may allow to detect an object in large visual environments at little cost. This should be particularly beneficial when the computing resources are under constraint, such as for drones or mobile robots. 

One limit of our model is the simplicity of the generative model for inputs. This simplicity was important to fully assess the possibility and robustness of implementing a ``where'' module which predicts the map of accuracies from the (degraded) log-polar transformed retinal image.
In future perspectives, more elaborate image categorization, such as the ones performed on natural images using deep convolutional nets, could be considered by replacing the current ``what'' module with a more elaborate one. 
%By preserving a probabilistic interpretation in bio-realistic action selection, 


Finally, our model relies on a strong idealization, assuming the presence of an unique target. The presence of many targets in a scene should be addressed, which amounts to sequentially select targets, in combination with implementing an inhibition of return mechanism. 
%Moreover, inhibition of return mechanism could envisioned by preserving a probabilistic interpretation in bio-realistic action selection. 
This would generate more realistic visual scan-paths over images. %This could be used to provide realistic priors over action selection maps.  
%In particular, identified regions of interest may then be compared with the baseline bottom-up approaches, such as the low-level feature-based saliency maps~\citep{Itti01}. 
Actual visual scan path over images could also be used to provide priors over action selection maps that should improve realism.  %
%It may indeed be possible to consider , and 
Identified regions of interest may then be compared with the baseline bottom-up approaches, such as the low-level feature-based saliency maps~\citep{Itti01}. 
Maximizing the Information Gain over multiple targets needs to be envisioned with a more refined probabilistic framework, including mutual exclusion over overt and covert targets. How the brain may combine and integrate these various probabilities is still an open question, that amounts to the fundamental binding problem. %: How is it possible to meaningfully combine independently extracted features.



% !TEX root = paper.tex
% !TEX encoding = UTF-8 Unicode
% -*- coding: UTF-8; -*-
% vim: set fenc=utf-8
% !TEX spellcheck = en-US
\section{Methods}
In this study, the visual scene is made of an unknown target at a random position and a noisy background (see Figure~\ref{fig:intro}). An agent controls a focal visual sensor that can move over the visual scene through saccades. In the implementation of such networks, we will follow the simplifying assumption that there is a separation between the inferences of  position and category in two respective pathways, namely the ``What'' and the ``Where'' pathways. The ``What'' pathway will be given from the literature and to test the validity of our hypothesis, it is necessary to find at least one function implementing the ``where'' network and that would be able to find the position of an object knowing only the degraded retinal image. Here, we describe the methods that we will follow to find that function, from the generative models (first external and then internal) to the actual implementation of that ``where'' pathway knowing a fixed ``what'' pathway. %
%There are however many shortcuts allowing to render the calculation amenable. This includes (i) sparse encoding, (ii) approximate inference through model separation, and (iii) sampling-based metric training. 
%------------------------------%
%: see Figure~\ref{fig:methods}
\begin{figure}[t!]%%[p!]
\centering{\includegraphics[width=\linewidth]{fig_intro}}%{fig_methods}}
\caption{
{\bf Methods for simulating active vision}:
\A We first define the model which generates images. It is composed of different random processes: one choosing a sample image from the MNIST database (of size $28\times 28$) and placing it at a random position within the circular mask on the $128\times 128$ display. Then, this image is rectified and multiplied by a contrast factor and finally embedded in a natural-like noise with characteristics its contrast, mean spatial frequency and bandwidth~\citep{Sanz12}. %
\B The full-sized images are transformed into a retinal image which will be fed to the ``where'' pathway. This is implemented by a bank of filters whose centers are centered of a log-polar grid and whose radius increases proportionally to eccentricity. Crucially, a similar transform is used to compute the accuracy of each hypothetical saccade, as represented by the collicular map. %
\C The ``where'' pathway is implemented by a three-layered neural network consisting of the retinal input, two hidden layers with $1000$ units each and a collicular output. Each unit is associated with a ReLU non-linearity. To learn to associate the output of the network with the ground truth, supervised training is performed using back-propagation with a binary cross entropy loss which measure the distance between both distributions. The network learns in about 20 epochs as shown by the decrease of the loss function. Overlaid is the associated accuracy of the full active agent. This is computed by classifying the foveal image using the ``what'' pathway, after centering the gaze using the result of the ``where pathway''. This shows a gradual increase in accuracy from the baseline ($10\%$) to approximately an average of $X80.0X\%$. %
}%
\end{figure}%
%%------------------------------%

\subsection{Exteroceptive Generative model}
It is first necessary to quantitatively define the generative model for input display images as shown first in Figure~\ref{fig:intro}-A (\DIS ) and implemented in see Figure~\ref{fig:methods}-A. 

\paragraph{Targets.} Following a common hypothesis regarding active vision, visual scenes will consist of a single visual object of interest. We will use the MNIST database of handwritten digits introduced by~\citep{Lecun1998} as classification solutions (``what'' pathway) abound for this class of targets and that we are here focused on the problem of localization (``where'' pathway). Samples are drawn from a database of $60000$ grayscale $28\times 28$ pixels images. 
\paragraph{Full-scale images.} For each sample, we may now draw a random position in a full-scale image of $128\times 128$. To enforce isotropic saccades, we define a centered circular mask covering the image (of radius $64$ pixels) and the position is such that the embedded sample fits entirely into that circular mask.
\paragraph{Background noise setting. } To provide with a realistic background noise, we generated synthetic textures~\citep{Sanz12} using a third random process. These textured images are of the same size of the full-image. These static images are designed to fit well with the statistics of natural images. We chose an isotropic setting where textures are characterized by solely two parameters. One controls the median spatial frequency $sf_0$ of this noise, while the other controls the bandwidth around this central spatial frequency. Finally, these can be considered as band-pass filtered images of random white noise. Finally, these images are rectified to have a normalized contrast.
\paragraph{Adding signal and noise. } Finally, both the noise and the target image are merged into a single image. We have used two different strategies. In a first strategy emulating a transparent association, we computed the average luminance at each pixel, while in a second strategy emulating an opaque association, we choose for each pixel the maximal value.
The quantitative difference were tested in simulations, but proved to have a marginal importance.
\subsection{Interoceptive generative model}


\paragraph{Foveal vision and the ``what'' pathway}
First, foveal vision is defined as the $28\times 28$ pixels image centered at the point of fixation (see dashed red box in Figure~\ref{fig:intro}-C). This image is then directly passed to the agent's visual categorical pathway (the ``What'' pathway). This is realized by the known ``LeNet'' classifier~\citep{Lecun1998}, that processes the $28 \times 28$ central pixels to identify the target category. Such a network is provided by the pyTorch library~\citep{Paszke17}, and consists of a 3-layered Convolutional Neural Network. It is trained over the (centered) MNIST database after approx $20$ training epochs. Input images are rectified (with a mean and standard deviation of respectively $0.1307$ and $0.3081$). The network outputs a vector representing the probability of detecting each of the $10$ digits. When taking the maximum probability, it achieves an average $98.7\%$ accuracy on a test dataset~\citep{Lecun1998}. % 

\paragraph{Retinal transform: Peripheral vision and log Polar encoding}
% >>> Laurent is here <<<
First, both the visual features and the expected target position may to be expressed in log-polar coordinates. On the primary visual side, local visual features (first and second order orientation filters) are radially organized around the center of fixation, with small and tightened receptive fields at the center and more large and scarce receptive fields at the periphery. The issued observation vector $\boldsymbol{x}$ compresses the original image by about 90\%, with high spatial frequencies preserved at the center and only low spatial frequencies conserved at the periphery.

The full-sized images are transformed into a retinal image which will be fed to the ``where'' pathway. This is implemented by a bank of filters whose centers are centered of a log-polar grid and whose radius increases proportionally to eccentricity. Crucially, a similar transform is used to compute the accuracy of each hypothetical saccade, as represented by the collicular map. %



\paragraph{Collicular representation: Metric training}

 In particular, we could also evaluate after this training phase the accuracy map of the classifier knowing a translational shift imposed to the input image. 


%The target accuracy map is also organized radially in a log-polar fashion, making the target position estimate more precise at the center and fuzzier at the periphery. This modeling choice is reminiscent of the approximate log-polar organization of the superior colliculus (SC) motor map {\bf[TODO:REF]}.
%This retinotopic organization is preserved along the visuo-motor pathway as expected from observations {\bf[TODO:REF]}.


%Though the effect of action is too complex to be inferred from a generative model, we assume here that it is trained by sampling, i.e. by "trial and error".

%The target accuracy map is also organized radially in a log-polar fashion, making the target position estimate more precise at the center and fuzzier at the periphery. This modeling choice is reminiscent of the approximate log-polar organization of the superior colliculus (SC) motor map {\bf[TODO:REF]}.
%This retinotopic organization is preserved along the visuo-motor pathway as expected from observations {\bf[TODO:REF]}.

 This central classifier displays a high accuracy at the center, and a fast decreasing accuracy with target eccentricity, as shown in figure \ref{fig:results}-D. In contrast, the visual orientation pathway (the ``Where'' pathway) takes the full visual field into account in order to tell whether a target is present at the different peripheral locations, in order to monitor future saccades.



%Though the effect of action is too complex to be inferred from a generative model, we assume here that it is trained by sampling, i.e. by "trial and error".


%<<<<<<< HEAD
%A second simplifying assumption is that the putative effect of a saccade should be condensed in a single number, the \emph{accuracy}, that is a statistics over the (scene understanding) benefit obtained from past saccades in the same context, independently of the identity of the visual objects. In detail, the primary visual information should be transformed so as to predict how accurate the categorical classifier will be after the saccade is carried out~\citep{Dauce18}. The set of all possible saccade predictions should form an \emph{accuracy map}.
%An accuracy map abstracts here a full sequence of operations, including (i) an initial visual examination, followed by (ii) a decision, (iii) a saccade realization and a (iv) second visual examination that should finally (v) determine the category of the target. 
%It should be mostly organized radially, preserving the initial retinotopic organization, with high predicted accuracies reflecting a high probability of target presence at given locations. 
%Such a \emph{predictive accuracy map} is assumed to be the cornerstone of a realistic saccade-based vision system, with action selection (motor map) overlaying the accuracy map through a winner-takes-all mechanism (as thought to be done in the superior colliculus). Of course, each different initial visual field comes with a different accuracy map (essentially conveying information about the target retinotopic position).
%Our main argument is that such an accuracy map is trainable in a rather straightforward way, through trials and errors, by actuating saccades after processing the visual input, and taking the final classification success or failure as a teaching signal. 
%\fi
%=======
%<<<<<<< HEAD

Active inference assumes a hidden emitter $e$, which is known indirectly through its effects on the sensor, that obey to a generative process : $x\sim p(X|e)$. The real emitter state $e$ being hidden, a parametric model $\theta$ is assumed to allow estimate the cause of the current visual field through model inversion thanks to Bayes formula, in short:
$$p(E|x) \propto p(x|E;\theta)$$
with $x$ the visual field in our case. Assume now that the cause $e$ of the visual field splits in two (independent) components, namely $e = (u,y)$ with $u$ the body posture (in our case the gaze orientation) and $y$ the object shape (or object identity). Assume also that a set of motor commands $A = \{..., a, ...\}$ may control the body posture, but not the object's identity, so that $y$ is invariant to $a$.
Then, before taking a decision, the consequence of every saccade should be analyzed  through model inversion \emph{over the future observations}, that is predicting the effect of every $a$ in $A$ to choose the action that may optimizes future inferences. The benefit of each $a$ is quantified through a certain metric (future accuracy, future posterior entropy, future variational free energy, ...), that depend on the current inference $p(U,Y|x)$. Each saccade $a$ is thus expected to provide a new visual sample from a given scene statistics, which may increase the understanding of the scene (here the target position and category). However, estimating the effect of every action over the range of every possible object shapes and body postures is combinatorially hard, even in simple cases such as vision, and thus infeasible in practice. 


%Though the effect of action is too complex to be inferred from a generative model, we assume here that it is trained by sampling, i.e. by "trial and error".

There are however many shortcuts allowing to render the calculation amenable. This includes (i) sparse encoding, (ii) approximate inference through model separation, and (iii) sampling-based metric training. 

\paragraph{Sparse encoding}
First, both the visual features and the expected target position may to be expressed in log-polar coordinates. On the primary visual side, local visual features (first and second order orientation filters) are radially organized around the center of fixation, with small and tightened receptive fields at the center and more large and scarce receptive fields at the periphery. The issued observation vector $\boldsymbol{x}$ compresses the original image by about 90\%, with high spatial frequencies preserved at the center and only low spatial frequencies conserved at the periphery.

\paragraph{Approximate inference}
Second, we %start as in~\citep{Friston12} by a probabilistic formulation, and 
use the fundamental hypothesis outlined in Figure~\ref{fig:intro}: the position of an object is independent from its category.  This allows considering $u$ and $y$ being independently inferred from the current visual field, i.e $p(U,Y|x) = p(U|x) p(Y|x)$. This property is strictly true in our setting and is very generic in vision for simple classes (such as digits) and simple displays (but see~\citep{Vo12} for more complex visual scene grammars). 
From this independence hypothesis, we may separate both inferences (identification vs localization) in two separate pathways with different morphologies (respectively foveal and peripheral). Note that from the retinotopic projection of the visual information, this independence is conditional on action: both pathways should update their beliefs upon decisions made in each respective pathway {\bf (??)}.
%A first simplifying assumption is a separation of the position and category inferences in two separate pathways, namely the ``What'' and the ``Where'' pathways.
The agent visual categorical pathway (the ``What'' pathway) is supposed to be realized by the known ``LeNet'' classifier~\citep{Lecun1998}, that processes the $28 \times 28$ central pixels to identify the target category (see dashed red boxes in  Figure~\ref{fig:intro}-C). This central classifier displays a high accuracy at the center, and a fast decreasing accuracy with target eccentricity, as shown in figure \ref{fig:results}-D. In contrast, the visual orientation pathway (the ``Where'' pathway) takes the full visual field into account in order to tell whether a target is present at the different peripheral locations, in order to monitor future saccades.

\paragraph{Metric training}
%=======
%>>>>>>> 10eba6746a264a8f6141953d7948057e5093489d
Third, the putative effect of every saccade should be condensed in a single number, the \emph{accuracy}, that is the expected benefit of issuing saccade $a$ %regarding the target identity, both assuming $p(U|\boldsymbol{x})$ and $p(Y|\boldsymbol{x})$ 
from the current observation. Taking $a$ a possible saccade and $\tilde{\boldsymbol{x}}$ the corresponding future visual field, the result of the categorical classifier over $\tilde{\boldsymbol{x}}$ can either be correct (1) or incorrect (0). 
If this experiment is repeated many times over many visual scenes, the probability of correctly classifying the future visual field $\tilde{\boldsymbol{x}}$ after a saccade $a$ forms a probability, i.e. a number between 0 and 1, that reflects the proportion of correct and incorrect classifications.
% when issuing a saccade $a$ after seeing $\boldsymbol{x}$ (the initial visual field). 
It more or less corresponds to inferring the true target identity $\hat{y}$, i.e. $p(\hat{y}|\tilde{\boldsymbol{x}})$, including the update of the eye direction, that is a sample of the ``real'' generative process. Active inference needs either the current identity $y$ or the current eye direction $u$ to be readable from the present view, in order to effectively predict future inferences, through computationally intensive predictions.   
Instead of doing predictions from a generative model, better off is to form a statistics over the (scene understanding) benefit obtained from past saccades in the same context, that is forming an \emph{accuracy map} from the current view. This is the essence of \emph{sampling-based metric prediction}.

In detail, the primary visual field should be transformed so as to predict how accurate the categorical classifier will be after the saccade is carried out~\citep{Dauce18}. %The set of all possible saccade predictions should 
An accuracy map abstracts here a full sequence of operations, including ($i$) an initial visual examination, followed by ($ii$) a decision, ($iii$) a saccade realization and ($iv$) a second visual examination that should finally ($v$) determine the category of the target.
It should be mostly organized radially, preserving the initial retinotopic organization, with high predicted accuracies reflecting a high probability of target presence at given locations.
Such a \emph{predictive accuracy map} is assumed to be the core of a realistic saccade-based vision system, with action selection (motor map) overlaying the accuracy map through a winner-takes-all mechanism (as thought to be done in the superior colliculus). Of course, each different initial visual field comes with a different accuracy map (essentially conveying information about the target retinotopic position).
Our main argument is that such an accuracy map is trainable in a rather straightforward way, through trials and errors, by actuating saccades after processing the visual input, and taking the final classification success or failure as a teaching signal.

\subsection{Implementing the ``where'' pathway}

This fundamental hypothesis was outlined in Figure~\ref{fig:intro}: the position $u$ of an object is independent from its category (identity) $y$. This allows considering $u$ and $y$ being independently inferred from the current visual field, i.e $p(U,Y|x) = p(U|x) p(Y|x)$. This property is strictly true in our setting and is very generic in vision for simple classes (such as digits) and simple displays (but see~\citep{Vo12} for more complex visual scene grammars). 
From this independence hypothesis, we may separate both inferences (identification vs localization) in two separate pathways with different morphologies (respectively foveal and peripheral). Note that from the retinotopic projection of the visual information, this independence is conditional on action: both pathways should update their beliefs upon decisions made in each respective pathway.
To check this hypothesis, we train a multi-layer neural network to predict accuracy maps from a retinotopic log-polar encoding of the visual image. 


\paragraph{Approximate inference}
Active inference assumes a hidden emitter $e$, which is known indirectly through its effects on the sensor, that obey to a generative process : $x\sim p(X|e)$. The real emitter state $e$ being hidden, a parametric model $\theta$ is assumed to allow estimate the cause of the current visual field through model inversion thanks to Bayes formula, in short:
$$p(E|x) \propto p(x|E;\theta)$$
with $x$ the visual field in our case. Assume now that the cause $e$ of the visual field splits in two (independent) components, namely $e = (u,y)$ with $u$ the body posture (in our case the gaze orientation) and $y$ the object shape (or object identity). Assume also that a set of motor commands $A = \{..., a, ...\}$ may control the body posture, but not the object's identity, so that $y$ is invariant to $a$.
Then, before taking a decision, the consequence of every saccade should be analyzed through model inversion \emph{over the future observations}, that is predicting the effect of every $a$ in $A$ to choose the action that may optimizes future inferences. The benefit of each $a$ is quantified through a certain metric (future accuracy, future posterior entropy, future variational free energy, ...), that depend on the current inference $p(U,Y|x)$. Each saccade $a$ is thus expected to provide a new visual sample from a given scene statistics, which may increase the understanding of the scene (here the target position and category). However, estimating the effect of every action over the range of every possible object shapes and body postures is combinatorially hard, even in simple cases such as vision, and thus infeasible in practice. 





\paragraph{Classifier training}
Modern parametric classifiers are composed of many layers (hence the term ``Deep Learning'') that can be trained through gradient descent over arbitrary input and output feature spaces {\bf [REF?]}. The ease of use of those tightly optimized training algorithms allow to quantify the difficulty of a task through training failure or success. Consider the visual features $\boldsymbol{x}$ as the input and a logpolar retinotopic vector $\boldsymbol{a}$ made of $n$ Bernouilli probabilities (success probabilities) as the output. A textured background noise is used at the input so that (i) the target statistics should barely differ from that of the background and (ii) a categorical classifier may not be trainable with the initial view as the input. A training set is made of randomly generated noisy images, with a low-contrast $28\times 28$ MNIST character  {\bf [REF?]} randomly positioned around the center of fixation (at maximum $30$ pixels from the center of fixation). The images are first whitened {\bf [REF?]}, and then transformed through $16384$ oriented log-polar filters radially organized around the center of fixation, with tight receptive fields at the center and large receptive fields at the periphery. The filters are organized in $10$ spatial eccentricity scales (respectively placed at around $2$, $3$, $4.5$, $6.5$, $9$, $13$, $18$, $26$, $36.5$ , and $51.3$ pixels from the center), and $16$ angular directions, allowing them to cover most of the original 128 $\times$ 128 image. Each visual input is accompanied with a corresponding accuracy map also organized radially through $10$ eccentricity scales and $16$ peripheral directions. The training is done in Pytorch {\bf [REF?]}. The parametric neural network is made of two fully connected hidden layers of size $1000$, with ReLu activation and $50 \%$ drop-out on the last layer. The network is trained over $600,000$ saccades on full-images, using the binary cross-entropy loss as the error signal, with a learning rate equal to $10^{-4}$. The training is done in about 3 hours on a laptop.




\nolinenumbers

% Either type in your references using
% \begin{thebibliography}{}
% \bibitem{}
% Text
% \end{thebibliography}
%
% or
%
% Compile your BiBTeX database using our plos2015.bst
% style file and paste the contents of your .bbl file
% here. See http://journals.plos.org/plosone/s/latex for
% step-by-step instructions.
%
\bibliographystyle{plos2015}

\bibliography{Bibliography}

\end{document}
