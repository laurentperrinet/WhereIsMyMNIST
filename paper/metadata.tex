% !TEX root = DauceAlbigesPerrinet2020.tex
%!TeX TS-program = pdflatex
%!TeX encoding = UTF-8 Unicode
%!TeX spellcheck = en-US
%!BIB TS-program = bibtex
% -*- coding: UTF-8; -*-
% vim: set fenc=utf-8
%: %%%%%%%%%%%%%%%%%%%%%%%%%%%%%%%%%%%%%%%%%%%%%%%%%%%%%%%%%%%%%%%%%%%%
%: METADATA
%: %%%%%%%%%%%%%%%%%%%%%%%%%%%%%%%%%%%%%%%%%%%%%%%%%%%%%%%%%%%%%%%%%%%%
\newcommand{\AuthorPA}{Pierre Albiges}
\newcommand{\AuthorED}{Emmanuel Dauc\'e}%
\newcommand{\AuthorLP}{Laurent Perrinet}%
\newcommand{\Address}{Institut de Neurosciences de la Timone, CNRS/Aix-Marseille Universit\'e, France}%
\newcommand{\WebsiteLP}{https://laurentperrinet.github.io/}%
\newcommand{\EmailLP}{Laurent.Perrinet@univ-amu.fr}%
\newcommand{\orcidLP}{0000-0002-9536-010X}%
\newcommand{\orcidED}{0000-0001-6596-8168}%
\newcommand{\Keywords}{Object detection \and Active Inference \and Visual search \and Visuomotor control \and Deep Learning}
\newcommand{\Title}{
A dual foveal-peripheral visual processing model implements efficient saccade selection
}
\newcommand{\Acknowledgments}{ TODO:  FRM ....... RIck + Karl + Laurent Madelain  - }
\newcommand{\Abstract}{
Visual search involves a dual task of localizing and categorizing an object in the visual field of view. We develop a visuomotor model that implements visual search as a focal accuracy-seeking policy, with the target position and category drawn independently within a common generative process. This independence allows to divide the visual processing in two pathways that respectively infer what to see and where to look, consistently with the anatomical separation between the ventral versus dorsal pathways. The ``What'' network is a classical deep learning categorization architecture. We use the dual principle to train a deep neural network architecture of the ``Where'' network with the foveal accuracy used as a monitoring signal for action selection. This allows in particular to interpret this network as a retinotopic action selection pathway, that drives the fovea toward the target in order to increase the central recognition accuracy. After training, the comparison of both networks accuracies amounts to either select a saccade or to keep the eye focused at the center, so as to identify the target. We test this on a simple task of finding digits in a large, cluttered image. A biomimetic log-polar treatment of the visual information implements the strong compression rate performed at the sensory level by retinotopic encoding, which is preserved up to the action selection. Simulation results first demonstrate that it is possible to learn this dual network. After training, this dual approach is also shown to provide ways to implement visual search in a sub-linear fashion, in contrast with mainstream computer vision.
}
\newcommand{\Precis}{
Separating visual processing into a What and a Where pathways provides a strategy to model visual search. We developed a deep-learning based computational model in which the comparison of predicted accuracies from both pathways allows for efficient saccade selection.
}
\newcommand{\AuthorSummary}{
A visual search task consists in extracting a scarce and specific visual information (the ``target'') from a large and cluttered visual display. In computer vision, this task is usually implemented by scanning all different possible target identities in parallel at all possible spatial positions, hence with strong computational load. The human visual system employs a different strategy, combining a foveated sensor with the capacity to rapidly move the center of fixation using saccades. Then, visual processing is separated in two specialized pathways. First, the ``where'' pathway conveys information about target position in peripheral space, independently of its category. On the other hand, the ``what'' pathway conveys information about the category of the target (independently of its position). This object recognition pathway is shown here to have an essential role, providing an ``accuracy drive'' that serves to guide the eye toward peripheral objects in order to increase the peripheral accuracy, much like in the ``actor/critic'' framework. Put together, all those principles are shown to provide ways toward both adaptive and resource-efficient visual processing systems.
}
%%%%%%%%%%%%%%%%%%%%%%%%%%%%%%%%%%%%%%%%%%%
